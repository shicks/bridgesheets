\documentclass[11pt]{article}

\title{Basics of Standard American Bidding}
\author{Stephen Hicks}

\def\C{$\clubsuit$}
\def\D{$\diamondsuit$}
\def\H{$\heartsuit$}
\def\S{$\spadesuit$}
\def\NT{\textsc{nt}}

\begin{document}
\maketitle
\section{Noncompetitive Auctions}
The basic premise behind noncompetitive auctions is to
efficiently identify a game, slam, or safe part-score.
As such, there are a few important numbers to keep in mind:\\

\begin{tabular}{|l|l|}\hline
26 & number of points needed for 3\NT, 4\H, 4\S\\\hline
29 & number of points needed for 5\C, 5\D\\\hline
\end{tabular}\\

In order to ensure that a possible game is not missed due to
lack of either partner opening, any hand with 13 or more points
should open.

Out of all the ways a pair of hands can split, more often than not,
there is at least one suit with an 8-card fit.  9-card fits are not as
common.  Out of possible 8-card fits, a 4-4 split is rather uncommon:
5-3 and 6-2 make up amuch larger fraction of the cases.  Thus, we
require 5-card suits to open majors, and it is likely that one partner
or the other will then bid this major suit.  Because hands don't
always have 5-card suits, we must allow minor openings with as few as
3 cards.

Whenever possible within the first couple rounds of bidding, it
is desirable for one partner to significantly narrow his point
range.  This makes the other partner ``captain'': he can now add
the small range of his partner's points to his own count and determine
the total strength of the partnership.  This partner is then responsible
for deciding which contract to ultimately play.  In order to effectively
communicate, it is prudent to break up the possible point ranges into
smaller categories.

\begin{minipage}[t]{0.5\columnwidth}
For the opener,\\
\begin{tabular}{|l|l|}\hline
13--16 & minimum opening\\\hline
17--19 & invitational\\\hline
20--22 & game strength\\\hline
23+    & slam strength\\\hline
\end{tabular}
\end{minipage}
\begin{minipage}[t]{0.5\columnwidth}
For the responder,\\
\begin{tabular}{|l|l|}\hline
6--10  & minimum response\\\hline
11--12 & invitational\\\hline
13--19 & game strength \\\hline
20+    & slam strength\\\hline
\end{tabular}
\end{minipage}

Note that all these ranges are based on the 26-point goal for
a major or no-trump contract.  If a minor contract is inevitable,
a small amount of shifting may be necessary.

First, we will look at the opener's ranges.  With 20 points or more,
even a minimal response with 6 from the partner is enough to bid
a game.  Opener should therefore make bids which do not allow passing
before game is reached.  With 17--19 points, we only need 7--9 points
from the responder to reach game (the top half of responder's minimum
range).  The responder's invitational range is similar, requiring
14--15 points from the opener (the top half of opener's minimum range).

Thus, if one partner has game-going strength, it is his duty to
force the bidding to game (or bid it directly if the correct game
is obvious).  If one partner has invitational strength, he must ask
the other for ``a little more'' than previously promised.  Much of the
following bidding structure is based on this premise, as well as the
(so far understated aspect of finding a fit in the first place).

\subsection{Opening bids}
There are several options for the opening bidder.  He may bid one of a
suit, showing a broad range of 13--22 points and a suit of 5 cards (or
3 in minor).  If he holds a balanced hand and between 15--17 points,
he may open 1\NT\ (or 2\NT\ with 20-22), thus telling his partner a very
specific range of points, making the responder captain, and opening
the door for a number of useful conventions.

With a weak distributional (long and short suits) hand, 
the opener may take up bidding space by making a slight gamble, 
preemptively bidding a long suit at a higher level, to communicate
both his weakness \emph{and} his distribution simultaneously.

Finally, with an amazing hand of 23 points, he opens with a conventional
artificial 2\C\ opening, which the partner may not pass, since it
says nothing about clubs, and since the hand is strong enough on its
own to reach the 3-level, even if partner has nothing.

[Brief aside: a \textbf{reverse} is a bid in which one partner bids a
new suit which is higher in rank (and at a higher level) than the suit
he began with.  This applies to the first couple rounds of bidding
while still searching for a fit.  A reverse promises a stronger hand
than a non-reversed order, and promises at least one more card in the
first suit than the second (6-5, 6-4, or 5-4, since a second new suit
must always have at least 4 cards).  Thus, 1\D-1\S-2\H\ and 1\C-1\H-2\D\ 
are reverses, while 1\C-1\D-1\S\ and 1\D-1\S-2\C\ are not.  1\C-1\D-2\S\ 
is a \emph{jump shift}, since 1\S\ could have been bid but was bypassed.]

Most of the time, an opening of one of a suit is most appropriate.
In these cases, we must decide which suit to open in.  Generally, one
should open the longest suit first, regardless of where the honors are
(although having honors in longer suits should increase slightly the
percieved value of the hand).  Possible exceptions exist with weaker
hands: 4 strong \D's should take precedence over 5 weak \C's with a
minimal (13--16) opening, since reversing (e.g. 1\C-1\H-2\D) would imply
invitational (17--19) strength.  Likewise, with a 5-6 split between
two suits and a minimal opening, the higher ranking suit should be
bid first to prevent a reverse (unless 6 \C's and 5 \S's, since \S\ can
be bid at the 1-level after either non-fitting suit response from partner).
With a 5-5 split, the higher ranking suit should \emph{always} be bid first,
since a reverse implies a \emph{longer} first suit (except with a \emph{very}
weak \C/\D\ opening).

\subsection{3rd and 4th seat openings}
When the first two seats pass, the dynamics change slightly.  The
third seat may open with a slightly weaker hand than he otherwise
would.  The typical rule for third seat openings is the ``rule of 20''
in which the player adds his high card points to his two longest
suits, and if it adds up to 20, he opens.  In fourth seat, the ``rule
of 15'' says that if the high card points plus the number of spades is
15 or more, he should open (since spades are favored in competitive
auctions, which are likely to ensue).  But since passing in fourth
seat leads to a redeal, he shouldn't open any hands he wouldn't want
to play.

When responding from a passed hand, the meanings change, since (a) there
is already an upper limit on the responder's point count (although it
may be more than 13 due to short-suit points if a fit is found), and
(b) the opener may or may not have full opening values.  Thus, responses
which show strong hands get new meanings.  And if the opener opened
with subminimum values, he should almost always pass the next bid.

\subsection{Responses to one of a suit}
There are a number of variations which may be implemented at this
point.  Each variation has its advantages and disadvantages.  The
traditional method is difficult for the responder to show an
invitational-strength fit, since a simple raise is minimal and a jump
raise is game-forcing.  The responder therefore must bid a different
suit and then return to the original suit in the next round.  The
convention of \emph{limit raises} was implemented to fill this gap,
but this comes at the cost of giving up the weak preemptive responses
and 2\NT\ responses.  Moreover, limit raises only applies to major
suits, so the roundabout method still applies to minor suit fits.  I
will there begin with no limit raises.

Since an opening of one of a suit is nearly unlimited in the point
range, the responder wants to narrow his point range if possible.  A
\textbf{simple raise} of opener's suit with 8 total trump shows
minimal responding points (6--10), while a \textbf{jump raise} shows 4
trump and maximal (13--16 or more) points, with the possibility of a
slam.  A \textbf{jump shift} shows maximal points and a different long
suit.  A \textbf{1NT} response shows a minimal response (6--10), and 
denies the possibility of bidding a major suit (with 4 cards) at the
1-level.  This is a last-resort response.  Finally, \textbf{2NT} shows
13--15 points and a balanced hand.

Barring any of these range-specific responses, the responder must bid
his longest suit at the cheapest possible level.  Note that with
invitational strength, this is the only reasonable option, even with a
fit.  Bidding at the 1-level (``1 over 1'' or \textbf{1/1}) requires a
4-card suit and 6 or more points (no maximum), while the 2-level (``2
over 1'' or \textbf{2/1}) requires a 4-card suit (or 5-card major (\H,
since one can always shift to \S\ at the one-level) and 10 or more
points (hence the 1\NT\ response, if the longest suit is unavailable
at the 1-level with a minimal hand).  Both of these responses are
forcing, and 2/1 promises another bid by responder, so it is forcing
twice.  Thus, the responder must be cautious that he \emph{has}
another reasonable bid, and if not, may undervalue his hand and bid
1\NT.

After a 1/1 (or 2/1) response, the responder is the captain, because the
opener will now clarify his hand a little more.  After any other
response, the opener is captain.

\subsubsection{Opener's rebid}

After a specific response, the opener must try to figure out where
to play the game.  If the opener's suit was raised, then the opener
can judge based on points where to play the game, possibly inviting
the responder to game with an invitational hand.  After a jump (which
is game-forcing, showing at least 26 total points), a fit may still
need to be found, and this can be accomplished by bidding other
suits, and eventually \NT\ if no fit is found.  Slams are also
a possibility after a jump, and one partner may invite a slam
with a still-stronger hand than he already promised.  After 1\NT,
the opener should assess whether game is still possible, and attempt
to either sign off, or find a fit.

However, the 1/1 and 2/1 responses are still the most common.  After a
1/1 response, the opener has a chance to narrow his point range very
specifically.  With a minimum opening (13--16), he may \textbf{raise}
either his own suit (with 6 trump), or his partner's suit (with 4-card
support), or bid \textbf{1NT} with a balanced hand and 13--14 points
and no biddable 4-card major suit (with 15--17 he would have opened
1\NT).  He may also bid a \textbf{\emph{new}} 4-card suit (as long as
it isn't a reverse), although this does not narrow the point range,
and the opener again becomes captain.

With invitational strength (17--19), the opener may bid \textbf{any
new} 4-card suit at the lowest level.  If it happens to be a \textbf{reverse},
then his strength is apparent, but if not, he is still not allowed to
jump shift, because that would imply 20 points.  He may also
\textbf{jump-raise} either his own suit (with 6 trump) or his partner's suit
(with 4 trump support).  A bid of \textbf{2NT} shows a balanced invitational
hand and stoppers (A, K-x, or maybe even Q-x-x) in all the unbid suits.

Finally, with game-forcing strength (20--22), he should
\textbf{reverse} or \textbf{jump shift} to force the bidding to game.
If a fit was already established, bidding game directly is advised if
no slam possibilities exist (\emph{note that a game bid is usually a sign
off}, unless the other partner has not fully revealed all his
strength).  Otherwise, with slam interest, he should avoid bidding
game and instead ask the partner about his holdings to determine if it
is possible (i.e. with Blackwood, or control-asking bids).

After a 2/1 response, the invitational (17--19) opening range is now
game-forcing, since $17+10>26$.  Thus, with 17 points, the opener may
make any jump rebid, and any non-jumping rebid suggests a weaker hand.
Any new suit may be bid at lowest level, although reverses should
still be reserved for 15 points or more.  Raising the responder's suit
(with 4-card support, to the 3-level) is dangerous, because the
responder \emph{must} bid again, and the contract will end up in game.
Thus, prefer a weaker bid if this might cause a problem.  2\NT\ shows
a balanced hand and stoppers in the unbid suits, and as always, any
new suit shows at least 4 cards.

\subsubsection{Responder's rebid}
At this point, it should be clear who is the captain.  If the captain
knows enough to bid a final contract (typical if a fit has been
found), he should do this.  If not, he must continue to give and get
more information.  If three suits have been bid and still no fit has
been found, then responder should begin to consider notrump as an
option.


\subsection{Responses to 1NT openings}
The 1\NT\ opening gives very specific information: a three-point
range for the opener, and only three possible distributions (4-3-3-3,
4-4-3-2, or 5-3-3-2), and denies a 5-card major.  The responder
can, with a balanced hand, invite or bid game in notrump.  With a 4-card
or better major suit, he can use conventional responses.

\paragraph{Stayman} is a bid of 2\C\ to ask the opener if he has a 
4-card major, and if so, to bid it.  2\D\ is a negative response.
If both partners have the same 4-card major, a fit has been found.
The 2\C\ bidder then has rebids to show 5-4 in the majors, to deny
a fit, or may raise to show a successful fit.

\paragraph{Jacoby Transfers} are a way to show a single long (5-card) major.
The responder bids 2\D\ or 2\H, one suit below his actual long suit,
and then the opener bids the long suit so that he is the declarer.
Since responder is captain, he may decide that game is unreachable and
sign off, or he may continue with forcing bids to press on to game.  With
a good fit, the opener may complete the transfer at the 3-level instead.

\paragraph{Busts} occur when the responder has minimal strength and is
convinced that a notrump contract will fail.  He will then bid 3 of a
long minor suit (since 2\C\ and 2\D\ are forcing conventions) as a
sign off.  Thus, to show a \emph{good} long minor suit, responder will
first bid Stayman (regardless of major suit holdings), and then raise
to 3 of his minor afterwards.  3\NT\ should still be an option, since
it is easier than 5 in a minor.


\subsection{Responses to strong 2\C\ opening}

An opening of 2\C\ shows 23 points or more and is almost always forcing
to game.  The only exceptions to this are when the opener (who is captain)
rebids bids 2\NT\ or raises his own suit one level.  Since the opening
is artificial and says nothing about clubs, it is forcing.  With a weak
hand (0--7 points), the responder bids 2\D\ as a first negative.  With
a stronger hand he bids his best suit or notrump at the lowest level,
giving the captain more information about both strength and shape.

The opener will rebid either notrump (2\NT\ opens Stayman, Jacoby, and
all the normal 2\NT responses, except it shows a slightly stronger hand)
or a suit at the lowest level (which is forcing).

If the responder previously bid the first negative, he must again cut
his possible point range in half by bidding a \emph{second negative}
with 0--4 points.  The second negative is shown by bidding the
cheapest minor, or 3\NT\ if the opener rebid 3\D.  Any other bid shows
5--7 points and gives information about a long suit (or 2\NT\ if balanced).

After a positive response, responder will bid game with a fit, show another
suit if no fit has been found, or will investigate slam possibilities with
more then a minimal positive response (10 points or more).  Exciting times.


\section{Limit Raises, Jacoby 2NT, and Splinters}

A conventional improvement on the system described above involves
shifting a few responses.  Since an invitation fit (11--12 points)
is difficult to show, requiring 1/1 or 2/1 and then later raising
to show more than 10 points, many have redefined the jump raise for
major suits to be invitational, rather than game-forcing.

Thus, after a major suit opening, 1\H, a 2\H\ response is minimal
(showing 3 trump, 6--10 points), and requires 17--19 points from opener
to invite 3\H, and 20 points to bid 4\H\ directly.  A response of
3\H is a limit raise, showing 11-12 points and 3 trump support
(sometimes 4, depending on whether another convention, 1\NT\ forcing,
is being used).

This raises the question of how a responder can show a \emph{good} fit.
Enter: Jacoby 2\NT\ and splinter bids.  A response of 2\NT\ to one
of a major is Jacoby and shows a 4-card fit and 13 or more points.
The responder is captain, and the opener is now asked to show more
about his hand (either side honors or singletons/voids).  Splinter
bids are double-jump shifts by the responder after one of a major, 
showing a 4-card fit, 13 or more points, and a singleton or void.
These are all game-forcing and investigate the possibility of a slam.

Several other conventions work well with these responses, but each
comes with the cost of giving up (or at best obfuscating) the
natural bids.

\section{Competitive Auctions}

\subsection{Overcalls}
When the opposing side has opened the bidding, the other side
must make an overcall or a takeout double to get into it.  These
all have different meanings than the ordinary openings.

Direct overcalls at the 1-level show 9--17 points and a 5-card suit.
Overcalls at the 2-level require a 6-card suit, and/or possibly better
values.  An overcall of 1\NT\ is identical to an opening, except it
requires a stopper in the enemy suit.

Weak jump overcalls may be made by skipping one or more levels.
The general rule is to count estimated tricks from a long trump
suit and to overbid by 2 tricks with unfavorable vulnerability,
3 tricks with equal vulnerability, or 3--5 tricks with
favorable vulnerability.  These are similar to the normal
preemptive openings.

A unique feature in competitive auctions is the cuebid.  After an
opponent has shown one suit, it is common to hold two of the
other three suits.  In this case, \textbf{Michaels cuebid} may be useful.
By articially bidding the right-hand opponent's suit (1\C-2\C),
it shows a two-suited (at least 5-5) hand.  After \C\ or \D,
it shows both majors.  After \H\ or \S, it shows the other
major and an \emph{unspecified} minor.  The responder may ask
for that minor by bidding 2\NT\ (and later 4\NT\ if no answer is
given).

Another response for two-suited hands is the \textbf{Unusual 2NT overcall}.
Overcalling a 1-level opening with 2\NT\ shows (5-5) in the two lower unbid
suits (noting that an artificial opening of 2\C\ does not count as bidding
it, so 2\C-2\NT\ shows \C\ and \D).  This may also be used after a strong 
1\NT\ opening, or a strong 2-bid, though it is no longer unusual after a
weak 2-bid.

\subsection{Takeout doubles}

A double for takeout means just that: the partner is expected to
take out the double by bidding over it.  It is another way to get
a foot into a competitive auction.  After a suit bid of 1, a takeout
double shows an openable hand (13 points or more) and decent length
in all three unbid suits.  Alternately, it could show 18 points or more
and any shape.  This is clarified by the opener's rebids.


\end{document}
