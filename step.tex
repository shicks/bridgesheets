% Begining of Preamble: use memoir class
\documentclass[oneside]{memoir}
\usepackage{colortbl,footnote,fullpage}%,pointtables}
\usepackage{bridge}
\usepackage[perpage,hang,symbol*]{footmisc}
\usepackage{calc,graphicx}
\renewcommand{\hangfootparskip}{0pt}
\usepackage{bridgememoir} % should take an argument ``extra'', 
% Defines conditionals:
%  @extratext - defaults to \@extratexttrue
%  colorbids  - defaults to \colorbidstrue
\usepackage[logonly]{trace}

\begin{document}
\traceoff

\title{Bridge Bidding: From The Ground Up}
\author{Stephen Hicks}

\maketitle

\vspace*{-1.5pc}
\tableofcontents*

\chapter*{Preface}
Before we can start talking about the wonderful game of bridge, I find it
necessary to explain a bit of how this document came to be, what I hope to
achieve with it, and how it should be used.

My goal for this book is to have a simple way to introduce bridge to people
who have played similar games such as spades, but have felt that the learning
curve of bridge was too much to take on all at once.  Thus, my primary goal
is to put together a mini-curriculum which can be introduced piece-by-piece
at a rate the learner is comfortable with.  At the same time, I mean to
build up a bidding system that is \textit{playable} at each intermediate stage.
In this way, an experienced player may play with a less experienced partner
and still be on the same page with regards to the bidding.

The first part is exactly this step-by-step system.  In each chapter,
I introduce several bidding concepts, giving what are hopefully compelling
and memorable reasons for why they are useful.  At the end of the chapter, I
present a mini ``convention page'' with a list summarizing the agreements
in play.  In some cases, I present two opposite treatments and leave the
choice to partnership agreement.  These are clearly marked by check-boxes
on the convention card, which should be photocopied and filled in.  In this
way, experienced players may see immediately what they can expect their
partners to be familiar with.  

I attempt to give numerous examples of each case.
A reader who has read (and played) through this part should end up
with a useful bidding system, based on the \textit{Standard American Yellow Card}
(SAYC) system developed by the American Contract Bridge League (ACBL).  This
system is widely played on many online bridge games, and is useful in clubs,
where strangers may end up as partners without having much chance to discuss
their own systems.  While the SAYC standard allows no choices regarding the
bidding---a decision which greatly helps for online and club games---I feel
that an established partnership can benefit greatly from deviating from such
standards where newer and more useful techniques are available.  Thus, I do
present several ``non-standard'' treatments and conventions mixed in with
the standard treatments in this part.  When this occurs, it is clearly
marked, and the corresponding convention page will have a check-box to show
that such recommended alternatives have been agreed upon.  Additionally,
I will try to thoroughly discuss the ramifications of using such a convention,
especially as regards to standard bids whose meanings must change.

The later parts are a number of topical embellishments on the system
presented in the first part which are not critical for basic bidding system.
A ``full'' convention page is given including all the discussed conventions
so that advanced partnerships may still keep track of their agreements.

\section*{What this book is not}
I do not plan to say much of anything about the \textit{play}, either as
declarer or defender.  I mention such issues only insofar as they relate
to bidding, such as briefly relating why a partnership might or might not
want to play from one hand or the other.

This book is also not a comprehensive encyclopedia of all possible bridge
conventions.  Such a task would be nearly impossible to do in the first
place, mainly because different partnerships play different conventions with
different subtle nuances.  Additionally, it would be out of date within a
couple years, as the bridge community is always developing new and better
conventions.  For such a project, please refer to the Bridge Wikia at 
\verb|http://bridge.wikia.org/|.

Finally, let me just say that I really have no qualification to write this
book.

\part{Step-by-step bidding}
\chapter{Rules}
Here I attempt to provide as brief as possible a summary of the rules of
bridge.  This summary is aimed at players who are familiar with games
in the spades/hearts/skat/euchre family.  If you are not familiar with any
of these games, I highly recommend learning one (preferably spades, as it
is the most similar) and gaining some familiarity before proceeding with
bridge.

\subsection{Auction}
The game consists of two partnerships of two players each, sitting
across from one another.  The deck is dealt out entirely, 13 cards to
a player.  At this point, the dealer has the first opportunity to make
a bid.  Bidding proceeds clockwise around the table.  Bids may be one
of 35 contract bids (a number from 1 to 7, followed by either a trump
suit or ``notrump''), pass, double, or redouble.  The contracts are
ranked in value, starting from 1\C, 1\D, 1\H, 1\S, 1\NT, 2\C, \ldots,
and reaching eventually 7\NT.  Any contract bid must be higher than
the previously-bid contract.  Any player may pass at any time.  Double
may only be bid if the last (non-pass) bid was a contract bid by an
opponent, and redouble is only allowed if the last (non-pass) bid was
a double by an opponent.  The auction ends as soon as three passes
occur in a row without any intervening bid (i.e. the last player to
make a non-pass bid is not allowed to bid again).

\subsection{Play}
Once the auction ends, the partnership who made the last contract bid
must play that contract.  The bid suit becomes the trump suit (unless
it was notrump), and the partnership must take a number of tricks
equal to the number bid plus six (half, rounded down).  The partner
who \textit{first} bid a contract of the eventual trump suit is known
as the \textit{declarer} and plays both hand.  The \textit{defender}
to the left of the declarer plays the opening lead.  At this point,
the declarer's partner turns his hand face up and becomes the
\textit{dummy}.  The declarer now plays both hands, and play continues
clockwise around the table.  All plays to a trick must follow the suit
that was led.  If a hand is out of that suit, any card may be played,
including a trump.  If a trump is present in the trick, the highest
trump wins the trick.  Otherwise, the highest card in the led suit is
the winner.  The winner of the trick takes the trick and leads to the
next trick from any card in his hand, including trump.  There are
\textit{no} requirements about trump having been broken before they
may be led.

\subsection{Scoring}
I will not go into the details of scoring here.  Suffice it to say
that points are won for each trick that is bid.  \C\ and \D\ are worth
20 points each, \H\ and \S\ 30 each, and \NT\ worth 40 for the first and
30 thereafter.  Once 100 points have been earned, the game is finished
and another game begins.  The partnership that wins the game possibly
gets a bonus.  Overtricks are scored as bonus points not counting
toward the 100 points for game.  Undertricks are penalized as bonus
points as well.  Doubled and redoubled contracts are more valuable in
all these respects.

The reason for explaining this is that the bids of 5\C, 5\D, 4\H, 4\S,
and 3\NT\ are all the lowest possible bids in each suit that are worth at
least 100 points.  As such, these are known as ``game bids'' and are to
be desired whenever possible.

\chapter{Terminology and Basics}
Before we can begin talking about any actual bidding, we must find a way
to distinguish between one hand and another.  What are the important aspects
of a hand that we want to convey to our partner?

The language of the bridge auction consists of 38 words:
\boxed{Allowed Bids: \P, 1\C, 1\D, 1\H, 1\S, 1\NT, 2\C, \ldots, 
       2\NT, 3\C, \ldots, 7\NT, \X, and \XX}
Moreover, there are heavy constraints on the sequencing of these bids.
Therefore, we do not have the luxury of telling our partner that we
have, card for card\footnote{nor would we want to if we could---this
  would give our opponents lots of information we'd rather them not
  have}.  Rather, we want to give a general idea of the
\textit{strength} of our hand, as well as the \textit{shape} of the
suit distribution.
As such, it help to have a quantitative way to understand strength and
shape.

\section{High card points}
We assign the following ``points'' for honors (note that these are \textit{not}
related to how a hand is scored---they are for self-evaluation purposes only):
\boxed{High Card Points (HCP): A---4, K---3, Q---2, J---1}
Additional modifications may be added as well\footnote{such as an
  extra point for all four aces, a penalty point for no aces, and not
  counting unprotected honors}.  Note that there are 40 HCP in the
deck and 10 in each suit.  

\section{High card points are not exact}
Certain combinations of honors are more or less useful than others.
In particular, a singleton king, or doubleton queen (Qx) is known as
\textit{unprotected} and is more likely to be worthless.  They should
therefore not generally be counted with the other high cards, or at least
the value should be dimished.  On the other hand, having many honors in
the same suit is more valuable then having all the same face cards scattered
over all the suits, and therefore the hand should be regarded as stronger.

Finally, the presence or absence of ``sub-honors'', or ``body'' cards, is
also important.  A hand with many 8s, 9s, and 10s is certainly stroger
than the same hand with 2s, 3s, and 4s instead, and should be treated as such.

\section{Distribution points}
High cards are not the only way to win tricks.  Having a particularly long
or short suit is also valuable, since the short suit allows \textit{ruffing}
trump and even the low values in the long suit may take tricks once all the
honors and trump are gone.  Thus, we also define distributional points.
These may be \textbf{either} long suit points or short suit points and are
to be added to the HCP whenever considering a \textit{suit contract} (other
than when the trump suit is short).
\boxed{%
  Long Suit Points: 1 point for each card after the fourth\\
  Short Suit Points: Void---3, Singleton---2, Doubleton---1}
Note that these measures are nearly equivalent, with the long suit points
valuing several balanced hands one point less.  Thus, \textbf{both types
of distributional points must \emph{not} be added on the same hand}.  Both are
presented here simply so that the reader may choose which he prefers to use.

On average, different level contracts require different amounts of points
between both partners.  These numbers are useful guidelines to remember,
but it should be stressed that with a good trump fit between the partners,
and a good split of trump between the opponents, contracts
can be made with much less, and with a poor fit and/or split,
a contract may fail even with an overabundance of points.
\traceon
\boxed{
  3\NT, 4-level suit contracts (including 4\H\ and 4\S)---26 points\\
  5-level suit contracts (including 5\C\ and 5\D)---29 points\\
  6-level contracts (``small slam'')---33 points\\
  7-level contracts (``grand slam'')---37 points}
\section{Fit and Split}
Discssion of different kinds of fit and what sort of (partnership)
splits are good and a little on how to tell this from the bidding.

\textit{Fit} is typically defined as when a partnership holds eight
or more cards in the trump suit.  This is most often split 5--3 or
4--4, but sometimes is more one-sided, as in 6--2.  These different
fits play out differently, but they are all fits nonetheless.  The
astute reader may note that if a partnership has as few as seven trump,
they still have a majority.  Often this is enough to make a contract,
although it's often more difficult because one opponent typically has
four of the outstanding trump, which is a lot.  This type of fit is
called a \textit{Moyesian fit} because it is poorer than the typical fit.

\textit{Split} refers to the distribution of trump (or other suits)
between the defenders.  Even if the declarer has an 8-card fit, it
is possible that one defender has all the remaining trump.  If the
declarer is 4--4, and one defender has 5 trump, then the declarer may
have a great difficulty making good use of these trump.  This is why,
once one opponent bids a suit naturally, it is generally a good idea
not to even investigate a possible fit in that suit, since chances are
that even if a fit is found, the split will be terrible.  There are
rare occasions when this is not the case, such as when a minor suit
opening shows only a 3-card suit.

\chapter{Basic Natural Openings and Responses}
We can now begin to discuss a rudimentary first attempt at a bidding
system.  As a first-order approximation, it is certainly not the most
usable system, but it will give the reader an entry point into actually
bidding and playing some hands.

\section{Strength requirements}
We don't want to stick our nose into the bidding unless there's a remote
chance that we can actually make the contract we bid (or one we can escape to).
Thus, we require the following
\boxed{
  13 or more points are required to open one of a suit.\\
  6 or more points are required to respond.}
Where did these numbers come from?  Because game generally requires about
26 points, we want to make sure that both partners don't pass if game is
a possibility.  By making the opening requirement 13 points, exactly half
of 26, it is certain that one partner or the other will open whenever there
is enough strength for game between them.  The number 6 is not quite as
obvious yet.  Later on, we will put an upper bound of about 21 on
this opening.  In this case, with less than 5 points, there is no hope for
game anyway, so there's not much reason to respond.

\section{What to open?}
If the bidding has passed around to you and you have the strength to open,
great!  It is now your \textit{duty} to your partner to tell him that by
opening the bidding.  But just what bid should you make?
\subsection{Five-card majors}
We pointed out above that \H\ and \S\ tricks are more valuable, and game
is easier in these suits.  These suits are known as \textit{major suits}.
Because major suits are more desirable than  minor suits (\C\ and \D),
we \textit{increase} the requirements for opening them:
\boxed{Opening 1\H\ or 1\S\ requires a five-card suit.}
The reason behind this is simple.  If majors were opened with a four-card
suit, then the responder needs a four-card suit to acknowledge that he
has a fit.  With five-card majors, responder only needs three to have
a fit.  What about when both partners have exactly four?  Then the opener
will open a minor suit and the responder will respond in a major suit, and
the 4--4 fit is just as easily discovered.
\subsection{Better minor}
Because we require five hearts or spades to open hearts or spades, we must
be careful that hands just short of this can open \textit{something}.  Consider,
in particular, hands with 4\S s, 4\H s, 3\D s, and 2\C s (hereafter denoted
4=4=3=2).  In this case, 1\H\ and 1\S\ are both forbidden.  We must therefore
allow opening (in this case) 1\D\ with only a three-card suit.  Likewise, if
the hand were 4=4=2=3, we would need to open 1\C\ with a three-card suit.
This is as bad as it gets, so we agree that
\boxed{
  Without a five-card major, open the longest minor suit, 1\C\ or 1\D.\\
  This may be as short as three cards.}
Furthermore, with 4--4 minors, open 1\D, and with 3--3 minors, open 1\C.
This treatment increases the chance of a 1\D\ opening being ``real''.
Although there is a possibility of a minor suit opening being based on
a three-card suit, it is generally best to assume a four-card suit until
told otherwise.

\subsection{With two long suits}
Sometimes hands are \textit{two-suited}, having 5 or more cards in each of
two suits.  In this case, there is a choice which suit to open.  If one
suit is longer, it should typically be opened first.  If one suit is a
major suit, it should be preferred over minor suits, particularly in
weaker hands.  Later we will introduce the concept of the \textit{reverse},
which will forbid certain second bids (``rebids'') from hands with less
than 17 points.  The upshot is that the highest-ranking suit should generally
be opened.  See REF? for more details on different situations which might
arise.

\section{Responding}

So your partner has gone out on a limb and opened the bidding.  And then
your right hand opponent passed, leaving you the chance to bid.  If you
have six or more points in your hand, it is now your solemn duty to tell
your partner, and you \textit{must} respond.  Remember, your partner may
be holding a monster hand and just needs the slightest encouragement from
you to bid game.

\subsection{How to respond?}
The type of response depends a lot on the type of opening.  If the opened
suit was a major, the responder's first priority is to confirm or deny an
eight-card fit (three-card support) in that major suit.  Without such a 
fit, responder should show a four-card suit (with preference to the other
major).

On the other hand, if the opening was in a minor suit, the first priority
is to find a 4--4 (or better) major suit fit.  Thus, responder should ignore
a possible minor suit fit if he has a four-card major to show.  Otherwise,
he confirms or denies a fit (four-card support, since minor suit openings
are shorter than major suit openings).

\boxed{After a major suit opening, raise the suit with three-card support.\\
\hskip2pc Otherwise, show a four-card suit.\\
After a minor suit opening, show a four card major suit if possible.\\
\hskip2pc Otherwise, raise the minor suit with four-card support.\\
\hskip2pc Otherwise, bid any other four-card suit or notrump.}

Now that we know what suit to bid, the question is what level to bid
it at.  If we always made the cheapest possible bid, then out partner
wouldn't be able to tell if we had 6 points or 26 points.  Therefore,
we should try to \textit{limit} ourselves to certain point ranges
whenever possible (specifically, when we have a fit---if we're still
looking for a fit, we don't have the luxury of using up so much
bidding room).  The general rule for this is to assume our partner has
the minimum strength he's promised, and then to bid what we can make
if we add our hand to that.  Since opener has promised 13 points, our
6 makes 19, which is about enough for 8 tricks if we have a fit.  So we
bid at the 2-level with anywhere from 6--10 points.  With 10--12 points,
we have 23--25 points as a partnership: just short of what we need for
a major suit game.  So a 3-level bid is what we want.  With 13--15, we
should be at the 4-level.  Note that for minor suits we should be wary
of raising at higher levels, in case the opener's suit is short.
\boxed{With 6--10 points and fit, bid the cheapest raise (2-level).\\
With 10--12 points and fit, bid a jump raise to the 3-level.\\
With 13 or more points and major suit fit, bid game (4-level).}

As far as new suits are concerned, we don't have the luxury of so fine a
graduation of point ranges.  Instead, we agree that with 6--12 points, we
will bid a new suit as cheaply as possible, and with 13 or more points,
we will bid a new suit, skipping a whole level of bidding.  This is known
as a \textit{jump shift}.
\boxed{\until{\sec4.1}With 6--12 points and a new suit, bid it as cheaply as possible.\\
\until{\sec4.1}With 13 or more points and a new suit, bid it one level higher.}

\section{Continuing the auction}
After the first two bids, if no fit is found, the opener should continue
to look for a fit by bidding any four-card suits.  This takes a bit of
thought, because the responder has not only shown which four-card suits
he \textit{has}, but also which he \textit{does not have}, by skipping over
any major suits.  So if the responder has already \textit{denied} a four-card
suit, there is no need to show it.  Doing so would then suggest a longer suit.

Without any suits to bid, notrump is also a possible bid.  It
generally promises at least a \textit{stopper} (protected honor) in
each suit that hasn't been named.  For point ranges, figure 20 to make 
1\NT\ and 23 to make 2\NT.

\section{Competition}
So far, we do not know how to deal with competition, that is, when one
partnership opens the bidding and then the other starts bidding over them.
% Recommendation: ...?

\chapter{Notrump Auctions}
So far we have only mentioned notrump bids in passing.    These bids provide
a very useful way of expressing \textit{balanced} hands.  We define
\boxed{A hand is \textit{balanced} if it contains at most one doubleton.\\
Balanced hand shapes are 4333, 4432, and 5332.}

Because balanced hands are always capable of opening in a suit, we have the
luxury of refining a 1\NT\ (and higher) opening to give much more detail
than a suit opening.
\boxed{A balanced hand with 15--17 high card points should open 1\NT.\\
A balanced hand with 20-22 high card points should open 2\NT.\\
\without{\sec A.1}A balanced hand with 25--27 high card points should open 3\NT.}



%\part{Extra stuff}
%LMO
%\chapter{Slams}


%\appendix
%\part{Appendices}
%\chapter{Glossary}
%AAA

\end{document}
