\documentclass{article}
\usepackage{colortbl,footnote,fullpage}%,pointtables}
\usepackage[perpage,hang,symbol*]{footmisc}
\renewcommand{\hangfootparskip}{0pt}

\makeatletter

% We need to do stuff like this to pull things out of the aux file safely.
\ifx\test\undef
  \def\test{placeholder}
\fi

\let\xa\expandafter

\def\cellcol#1#2{\multicolumn{1}{>{\columncolor{#1}}c}{#2}}
\let\mc\multicolumn

\def\cellcol#1#2{\multicolumn{1}{>{\columncolor{#1}}c}{#2}}
\def\mc#1#2{\multicolumn{#1}{c}{#2}}
\def\mcc#1#2#3{\multicolumn{#1}{>{\columncolor{#2}}c}{#3}}

\definecolor{lblue}{rgb}{.75,.8,1}
\definecolor{lred}{rgb}{1,.8,.8}
\definecolor{lgreen}{rgb}{.8,1,.8}
\definecolor{lorange}{rgb}{1,.9,.7}
\definecolor{lpurple}{rgb}{.95,.7,1}
\definecolor{lyellow}{rgb}{1,1,.6}
\definecolor{laqua}{rgb}{.65,.95,1}

% See if we can't get better colors (both values and names and meanings)
% Forcing
% Game Forcing
% Invitational
% Slam asking?
% Asking/telling?
% Captain?
% - Striped background for responses which are off after interference
%   (various kinds for off after double and/or overcall??)

\newcommand{\cb}[2]{\mcc#1{lblue}{#2}}
\newcommand{\cp}[2]{\mcc#1{lred}{#2}}
\newcommand{\cg}[2]{\mcc#1{lgreen}{#2}}
\newcommand{\co}[2]{\mcc#1{lorange}{#2}}
\newcommand{\cv}[2]{\mcc#1{lpurple}{#2}}
\newcommand{\cy}[2]{\mcc#1{lyellow}{#2}}

\def\C{$\clubsuit$}
\def\D{$\diamondsuit$}
\def\H{$\heartsuit$}
\def\S{$\spadesuit$}
\def\NT{\textsc{nt}}
\let\TeXto\to
\let\TeXge\ge
\let\TeXle\le
\def\to{$\TeXto$}
\def\lt{$<$}
\def\gt{$>$}
\def\le{$\TeXle$}
\def\ge{$\TeXge$}
\def\ltsim{$\lesssim$}
\def\gtsim{$\gtrsim$}

\makesavenoteenv{tabular}

\begin{document}
% Dimensions: bw=remaining width; bh,bd=box height and depth
%             corr=correction for colorbox padding
\newdimen\scb@bw\newdimen\scb@bh\newdimen\scb@bd\newdimen\scb@corr
% This recursively typesets the stripes, one at a time
\def\scb@stripes{\let\scb@next\scb@stripes
  % First make sure we aren't over the bounds
  \ifdim\scb@bw<0\p@\let\scb@next\relax\else
  % See if we need to draw a partial stripe
  \ifdim\scb@bw<\scb@sw\relax
    % Correct the width if so, and make this the last loop
    \advance\scb@corr-\scb@sw\relax\advance\scb@corr\scb@bw\let\scb@next\relax
  \fi
  % Subtract off sw (stripe width) and ww (white width) from the remainder
  \advance\scb@bw-\scb@sw\global\advance\scb@bw-\scb@ww
  % Draw the colorbox and then skip over the white - could be generalized
  \expandafter\colorbox\scb@col{\lower\scb@bd\vbox to \scb@bh{}\hskip\scb@corr}\hskip\scb@ww\fi
  \scb@next}
% Typesets the two boxes, one on top of the other
\def\scb@scb#1{%
  % Calculate the width of the colorbox padding
  \setbox0\hbox{\expandafter\colorbox\scb@col{}}\scb@corr\scb@sw\advance\scb@corr-\wd0%
  % Put everything into an hbox, calculate the width of the text
  \hbox{\setbox0\hbox{#1}\scb@bw\wd0\scb@bh\ht0\advance\scb@bh\dp0\scb@bd\dp0%
    % Draw the stripes, then set it to zero size
    \setbox0\hbox{\scb@stripes}\wd0=0\p@\ht0=0\p@\dp0=0\p@
    % Typeset both boxes
    \box0\hbox{#1}}}
% Pull off the optional widths
\newcommand{\scb@getwidths}[1][5pt]{\scb@parsewidths#1,#1,\@@\scb@scb}
% Parse the widths and stick into variables
\def\scb@parsewidths#1,#2,#3\@@{\def\scb@sw{#1}\def\scb@ww{#2}}
% Grab the color into col, then keep processing
\newcommand{\stripedcolorbox}[2][*]{\ifx#1*\relax\def\scb@col{{#2}}\else
  \def\scb@col{[#1]{#2}}\fi\scb@getwidths}

\iffalse
\test
\newcolumntype{C}{>{\phantom{m}}c<{\phantom{m}}}
\newcolumntype{W}{c<{\phantom{m}}>{\phantom{m}}c}
\newcolumntype{R}{c<{\phantom{m}}>{\phantom{n}}c<{\phantom{n}}>{\phantom{m}}c}

\begin{tabular}{lcccccccccc}
\hline
\cellcol{black}{} & \co1{0--6} & \cb1{6--8} & \cg2{8--10} & \cp1{10--12} & \cv1{13--14} & \cp1{15--16} & \cg1{17--19} & \cb1{20--22} & \co1{23+}\\\hline\hline
Opening && \co4{Weak 2} & \cb2{Min} & \cg1{Inv} & \cp1{Max} & \cv1{Strong 2}\\\hline
Response && \cb3{Weak} & \cg1{Inv} & \cp5{Strong}\\\hline\hline

Direct o/c &&&\phantom{r}& \cg7{Overcall}\\\hline
Takeout Dbl &&&&&&&& \cp3{Strong}\\\hline

Invit'l cuebid &&&&\phantom{r}& \cb1{Raise} & \cp5{Raise or Shift}\\\hline
Michaels && \co4{Weak} &&& \cp3{Strong}\\\hline
Unusual NT && \co4{Weak} &&& \cp3{Strong}\\\hline
\hline
\end{tabular}

\colorbox{lblue}{\vbox to 2 in{\hbox to 2in{Testing 12 13 14 15}\hbox to 2in{a b c d ef\hss}}}
% Could make i.e. \row[height=2pc]\cell[4]{...}\cell[3]{...}
% in terms of units set at the top..  But how to scale properly?
% Would like to make it as narrow as possible, to accomodate everything.
% We could accumulate box widths and then divide to keep a running total
% of the minimum length of the unit.

%\beginbidtable abc\endbidtable
\eject
\fi

\let\xa\expandafter
\newcount\c@a\newcount\c@b\newcount\c@c
\newif\ifplus
\def\@gobble@null\@null{}
\def\@maybe@next{\ifx\@foo\@null\let\@next\@gobble@null\fi\@next}
\def\@headings#1|{\doheading#1-.-\@end\let\@next\@headings\futurelet\@foo\@maybe@next}
\def\headings#1{\global\c@c\z@\@headings#1|\@null\\\hline\hline}
\def\stripplus#1+#2\@null{\ifx+#2\relax\plustrue\else\plusfalse\fi\c@a#1\relax}
\def\setupper#1#2\@null{\ifx.#1\relax\number\c@a\relax\c@b\c@a\relax\else\c@b#1#2\relax\number\c@a--\number\c@b\fi\ifplus+\fi}
\def\doheading#1-#2-#3\@end{%
  &%  \ifnum\c@c>\z@&\fi
  \stripplus#1+\@null % sets \ifplus and \c@a
  \setupper#2\@null   % sets \c@b and prints stuff
  \xa\xdef\csname a@\romannumeral\c@a\endcsname{\number\c@c}%
  \xa\xdef\csname b@\romannumeral\c@b\endcsname{\number\c@c}%
  \global\advance\c@c\@ne % next column...
}


\def\test{&1&2&3}

\def\getleft{\xa\c@a\csname a@\romannumeral\c@a\endcsname}
\def\getright{\xa\c@b\csname b@\romannumeral\c@b\endcsname}

\def\@@lign{%
  \t@b%  % This should work now...
  \let\@next\@@lign
  \ifnum\c@c<\c@a
    \global\advance\c@c\@ne
  \else
    \let\@next\relax
  \fi\@next}

\newcount\@left
\newcount\@cur

\def\t@b{&}
\def\@lign{%
  \global\@left\c@a % copy this value
  \global\advance\@left-\@cur % find out how many more we need to go.
  \ifnum\@left>\z@  % SOMEHOW, this counter is getting changed on me!!!
    \t@b\xa\multicolumn\xa{\number\@left}{c}{}\t@b%
  \else\t@b\fi
}

\def\@bid#1#2#3#4{%
\c@a#1\relax\getleft % \c@c stores the current position in the row...
\c@b#2\relax\getright\advance\c@b\@ne
\global\@cur\c@c\relax\global\c@c\c@b\relax % make sure to update it...
\global\advance\c@b-\c@a\relax
\@lign\mcc{\number\c@b}{#3}{#4}%
}
\def\resolverange#1-#2-#3\@null{{#1}{#2}}
\def\@@bid#1#2#3{% #1 is a range
\xa\@bid\resolverange#1-#1-\@null{#2}{#3}}

\newif\if@star
\newif\if@usingstar
\global\@usingstartrue
\def\bid{\futurelet\@foo\check@star}
\def\check@star{\let\@next\@nsbid
  \ifx\@foo*\relax
    \if@usingstar\global\@startrue\fi
%    \global\let\if@star\if@usingstar % bad syntax...
    \let\@next\@sbid
  \else
    \global\@starfalse
  \fi\@next}
\def\@sbid*{\@nsbid}
\def\@nsbid(#1,#2)#3{\@@bid{#1}{#2}{\if@star\else#3\fi}}

\def\clear@ranges{\count@40\let\@next\@clr@ranges\@next}
\def\@clr@ranges{\ifnum\count@<\z@\let\@next\relax\else
  \xa\global\xa\let\csname a@\romannumeral\count@\endcsname\undef
  \xa\global\xa\let\csname b@\romannumeral\count@\endcsname\undef
  \global\advance\count@\m@ne
\fi\@next}

\def\b@ptable{\begin{tabular}{lccccccccccccccccccccc}} % should be enough...
\def\e@ptable{\hline\end{tabular}\clear@ranges}

\def\gap{\\[-.6pc]}
\newcommand{\drop}[2][0.52pc]{\setbox0\hbox{\lower #1\hbox{#2}}\dp0=0pt\ht0=0pt\box0}
\newcommand{\cdrop}[2][0.52pc]{\if@usingstar\setbox0\hbox{\lower #1\hbox{#2}}\dp0=0pt\ht0=0pt\box0\else#2\fi}

\newenvironment{ptable}{\b@ptable}{\e@ptable}

\def\row{\global\c@c\z@}
\def\spanrow#1#2{\multicolumn{#1}{l}{#2}\c@c\m@ne\global\advance\c@c#1}
\def\ob#1#2{\@@bid{#1}{lorange}{#2}}
\def\yb#1#2{\@@bid{#1}{lyellow}{#2}}
\def\gb#1#2{\@@bid{#1}{lgreen}{#2}}
\def\bb#1#2{\@@bid{#1}{lblue}{#2}}
\def\pb#1#2{\@@bid{#1}{lred}{#2}}
\def\vb#1#2{\@@bid{#1}{lpurple}{#2}}
\def\ab#1#2{\@@bid{#1}{laqua}{#2}}

% There MUST be better ways to do this...
\def\oc#1#2{\@@bid{#1}{lorange}{}}
\def\yc#1#2{\@@bid{#1}{lyellow}{}}
\def\gc#1#2{\@@bid{#1}{lgreen}{}}
\def\bc#1#2{\@@bid{#1}{lblue}{}}
\def\pc#1#2{\@@bid{#1}{lred}{}}
\def\vc#1#2{\@@bid{#1}{lpurple}{}}
\def\ac#1#2{\@@bid{#1}{laqua}{}}

\def\ic{\futurelet\@foo\@ic}
\def\@ic{\let\@next\g@ic\ifx\@foo\egroup\let\@next\relax\fi\@next}
\def\g@ic#1{\futurelet\@foo\@ic}

\def\ditto#1{\raise 0.18pc\hbox to #1{\leaders\hrule\hfill\lower 0.4pc\hbox{\kern1ex''\kern0.6ex}\leaders\hrule\hfill}}

%\def\savenote#1#2{\begingroup\count@\value{footnote}\advance\count@\@ne\xdef#1{\the\count@}\endgroup\footnote{#2}}
%\def\savenote#1#2{\footnote{#2}\xdef#1{\value{footnote}}}

% Want some way to signify various properties - forcing, overcalls, etc...
\def\w#1{\def\@top{}\def\@bot{}\@w#1\@go}
\def\@w#1{\let\@next\@w\ifx#1\@go\relax\let\@next\@go\else
  \ifx#1o\relax\edef\@top{\@top o}\fi
  \ifx#1d\relax\edef\@top{\@top d}\fi
  \ifx#1i\relax\edef\@bot{\@bot i}\fi
  \ifx#1s\relax\edef\@bot{\@bot s}\fi
  \ifx#1f\relax\edef\@bot{\@bot f}\fi
  \ifx#1g\relax\edef\@bot{\@bot g}\fi
\fi\@next}
\def\@go{ ${}^{\mathrm{\tiny\@top\@bot}}$}%_{\mathrm{\tiny\@top}}$}

\section{Openings}

\begin{ptable}\headings{0-5|6-8|9-12|13|14|15|16|17|18|19|20|21|22|23+}
\row{balanced}
%  \[13-14,G]{...}
  \bid(13-14,lgreen){}
%  \bid(15-17,lblue){\vbox{\hbox{\vsize=.5pc\tiny 1NT}\hbox{\vsize=.5pc\colorbox{lgreen}{\tiny1/1,2/1}}}} % UGLY...
  \bid(15-17,lyellow){1NT}
  \bid(18-19,lgreen){}
  \bid(20-22,lred){2NT}
  \bid*(23,lpurple){2\C}\\%\hline
\row{openable suit\footnote{Opening a suit requires 5 of a major or 4 of a minor}}
  \bid*(13-22,lgreen){1\C\D\H\S}
  \bid*(23,lpurple){2\C}\\%\hline
\row{6 card suit}
  \bid(6-12,lblue){2\D\H\S}
  \bid(13-22,lgreen){1\C\D\H\S}
  \bid(23,lpurple){\cdrop[-.45pc]{2\C\w{f}}}\\%\hline
\row{7 card suit}
  \bid(0-8,lorange){\ge3\C\D\H\S}
  \bid*(9-12,lblue){2\D\H\S}
  \bid*(13-22,lgreen){1\C\D\H\S}
  \bid*(23,lpurple){2\C}\\
\spanrow{5}{solid minor, 2 outside stoppers\footnote{7-card minor (sometimes 6), usually all 3 outside suits stopped, never a singleton or void}}
  \bid(16-21,lyellow){Acol 3NT\w{f}}\\
\end{ptable}

%% Hack of colortbl:
%\def\@lf{\hbox{\vrule\hbox to 1pt{}}}
%\def\fake#1{\setbox0\hbox{#1}\wd0=0.5pt\ht0=0pt\dp0=0pt\box0}
%\def\fakes#1{\let\@next\relax\ifnum\count@>\z@\fake#1\advance\count@\m@ne\let\@next\fakes\fi\@next}
%\def\CT@@do@color{%
%  \global\let\CT@do@color\relax
%        \@tempdima\wd\z@
%        \advance\@tempdima\@tempdimb
%        \advance\@tempdima\@tempdimc
%        \kern-\@tempdimb
%        \leaders\hbox{\large\fake/\fake/\fake/\fake/\fake/\fake/\hskip 3.1pt}%\vrule
%%^^A                     \@height\p@\@depth\p@
%                \hskip\@tempdima\@plus  1fill
%        \kern-\@tempdimc
%        \hskip-\wd\z@ \@plus -1fill }

\section{Responses to 1 of a suit}
\textbf{Limit Major Raises}, \textbf{Jacoby 2NT},
\textbf{Splinter Bids}, \textbf{Weak Jump Shifts}, and \textbf{Inverted
Minor Raises}\\

%\begin{tabular}{cc}
%  abcdefg&hijklmnop\\
%  {\color{green}\leaders\hbox{/\hskip-3pt}\hfill}\hskip 0pt\@plus-1fill test\\
%\end{tabular}

\noindent\begin{ptable}\headings{2-6|6-10|11-12|13+}
\row{three-card major suit fit}
  \bid*(6-10,lblue){single raise (2M)}
  \bid*(11-12,lgreen){limit raise (3M)}
  \bid(13,laqua){1/1 or 2/1}\\
\row{four-card major suit fit}
  \bid(6-10,lblue){single raise (2M)}
  \bid(11-12,lgreen){limit raise (3M)}
%  \bid(6-12,white){\ditto{2in}}
  \bid(13,lpurple){Jacoby 2NT\w{go}}\\
%\row{\ldots\ + 1$^{\underline{\mathrm{tn}}}$/void}
\row{\ldots\ + singleton/void}
  \bid*(6-10,lblue){single raise (2M)}
  \bid*(11-12,lgreen){limit raise (3M)}
%  \bid(6-12,white){\ditto{2in}}
  \bid(13,lyellow){splinter bid\footnote{Splinter bid is a double-jump (non-game) bid in new suit, showing a singleton.}\w{g}}\\\\[-.6pc]
%\row{5 trump fit}
%  \bid(7-10){preempt to game (4M)}
\row{four-card minor suit fit}
  \bid(6-10,lblue){jump raise (3m)\footnote{Prefer 1NT with a bare 10-point raise}}
  \bid(11-13,lgreen){single raise (2m)}\\\\[-.6pc]
\row{\drop{new suit\footnote{All new suits require at least 4 cards.
After a minor suit opening, prefer bidding 4-card suits up the line
at the 1-level.  2\H\ requires a five-card suit.  Weak jump preempts
require roughly 6 cards at the 2-level and 7 at the 3-level.}}}
  \bid*(2-6,lorange){jump shift}
  \bid(6-10,lblue){1NT\footnote{\label{ntresp}1NT response/rebid denies a biddable 1-level major}}
  \bid(11-13,lgreen){2/1}\\
\row{}%\row{\phantom{m}\ditto{0.6in}}
  \bid(2-6,lorange){\cdrop[-.45pc]{jump shift}}
  \bid(6-13,laqua){1/1}\\%\\[-.6pc]
%\row{six-card suit}
%  \bid(2-6,lorange){jump shift\savenote{\weakshift}{Weak jump shifts 
%      at the three level \emph{must} have a slightly longer suit, 
%      and \emph{may} show an extra point or two}}
%  \bid(6-13,white){\ditto{3in}}\\
%four-card suit%\footnote{2\H requires 5 cards}
%(5 to bid 2\H)&\co1{preempt\footnote{double jump shift requires 7-card suit}}&\cy1{1NT}&\cg4{2/1}\\
\end{ptable}

\section{Responses to 1NT}
\begin{ptable}\headings{0-8|8-10|11-12|13-15|16-17|18+}
\row{(semi)balanced}
  \bid(8-10,lblue){2NT}
  \bid(11-15,lred){3NT}
  \bid(16-17,lpurple){4NT}
  \bid(18,lyellow){6NT}\\
\row{five-card major}
\iffalse % Texas Transfers
  \bid(0-10,lgreen){Jacoby transfer 2\D\H}
  \bid(11-15,lblue){Texas transfer 4\D\H}
  \bid(16-18,lgreen){2\D\H}\\
\else
  \bid(0-18,lgreen){Jacoby transfer 2\D\H}\\
\fi
\row{four-card major}
  \bid(0-8,lblue){2\C--Pass\footnote{only pass Stayman response with 4441 shape}}
  \bid*(8-18,lyellow){Stayman 2\C}\\
\row{5 or 6-card minor}
  \bid(0-8,lorange){3\C\D\footnote{minor suit bust requires 6 cards}}
  \bid(11-18,lyellow){\cdrop[-.45pc]{Stayman 2\C}}\\
\end{ptable}

\section{Responses to 2NT}
\begin{ptable}\headings{0-3|4|5-7|8-10|11+}
\row{(semi)balanced}
  \bid(5-10,lblue){3NT}
  \bid(11,lred){4NT}\\
\row{five-card major}
  \bid(0-11,lgreen){Jacoby transfer 3\D\H}\\
\row{four-card major}
  \bid(4-11,lyellow){Stayman 3\C}\\
\row{5 or 6-card minor}
%  \bid(0-7,lorange){3\C\D (w/ 6+ cards)}
  \bid*(8-11,lyellow){Stayman 3\C}\\
\end{ptable}

\section{Responses to Strong 2\C}
\begin{ptable}\headings{0-7|8+}
\row{long suit}\bid*(0-7,lorange){2\D}\bid(8,lpurple){2M, 3m}\\
\row{balanced}\bid(0-7,lorange){\cdrop[-.45pc]{2\D}}\bid(8,lred){2NT}\\
\end{ptable}

\subsection{Rebids after Negative 2\D}
\begin{ptable}\headings{23-24|25-27|28-30|31-32|33+}
\row{5-card suit\footnote{strong 4-card suit if 4441}}\bid(23-33,lyellow){2M, 3m}\\
\row{balanced}\bid(23-24,lblue){2NT}

\end{ptable}


\section{Responses to Weak 2}
\begin{ptable}\headings{0-4|5+}
\row{Support}\bid(0-5,lgreen){raise}\\
\end{ptable}

% Organization: Do we want to do all the responses first,
% and then the rebids, or do we want to separate out the
% various types of auctions and show all responses and rebids
% from there?  Probably the latter...

\section{Opener's rebids after 1/1}
\begin{tabular}{lcccc}
&\mc2{13--16}&\mc1{17--19}&\mc1{20--22}\\\hline\hline
fit (4 trump)&\cb2{single raise}&\cg1{jump raise}&\cv1{reverse, jump shift}\\\hline
balanced & \cy1{1NT\footref{ntresp}%{1NT response after 1/1 gaurantees no 4-card major at 1-level}%
}&&\cp1{2NT (w/stops)}&\\\hline
4-card unbid suit &\cg3{non-reverse}&\cv1{jump shift}\\
                &&&\cp2{reverse}\\\hline
6-card original suit &\cy2{rebid suit}&\cb1{jump rebid}\\\hline
\end{tabular}

\subsection{1/1 bidder's rebids after opener's single raise}
6--10 points: Pass\\
11--12 points: Invite game in suit only at 3-level.  Invite game in either
suit or notrump with 2NT.  An unbid suit invites game in agreed-upon suit,
but shows side honors or length.  3 of opener's original suit shows only
4-cards in own suit, but good fit in opener's suit, invites game.\\
13+ points: Bid game, or bid or investigate slam

\subsection{1/1 bidder's rebids after opener rebids 1NT (minimum)}
6--10 points: Scramble for a safe part-score (non-reverse, non-jump)\\
11--12 points: 2NT, or over/under-bid\\
13+ points: Reverse or jump bidding

\subsection{1/1 bidder's rebids after opener's nonjump rebid of original suit}
6--10 points: Pass, unless \le1 trump \emph{and} a 6-card suit to escape to\\
11--12 points: Show another (4-card) suit (forcing one round), or 2NT with
unbid suits stopped, or raise with as meager as 2-card major or 3-card minor 
support\\
13+ points: Bid game if known, or forcing new suit otherwise

\subsection{1/1 bidder's rebids after opener's non-reversing shift to unbid suit}
6 points: Pass\\
7--10 points: 1NT if balanced and no fit, or show preference for one of opener's suits at 2-level\\
11--12 points: (non-jumping) 2NT with last suit stopped, fourth suit (forcing),
invite game in one of opener's suits with non-jump to 3-level; jump from a 
passed hand\\
13+ points: jump
\subsubsection{1NT responder's rebids after opener's non-reversing shift}
Prefer second suit with 4-card support, bid 6-card suit at 2-level, or 7-card
suit at 3-level.

\subsection{1/1 bidder's rebids after opener's reverse}
6--8 points: 2NT with last stopper, or nonjump in either of first two suits\\
9+ points (game-forcing): fourth suit, raise third suit (4 trump), jump first suit, 3NT

\subsection{1/1 bidder's rebids after opener's jump bid}
\subsubsection{After opener's jump raise, jump rebid, or 2NT}
Pass with 6--7, bid suit or NT game with 8+.
\subsubsection{After opener's jump shift}
Bid naturally to show strength.  Game is required.

\section{Opener's rebids after 2/1}
% GRR...  these tables are a pain, especially for stupid systems
\begin{tabular}{lcccc}
&\mc1{13--14}&\mc1{15--16}&\mc1{17--19}&\mc1{20+}\\\hline\hline
fit (4 trump)&\cb2{single raise}&\cg1{jump raise}&\cv1{reverse, jump shift}\\\hline
balanced & \cy1{1NT}&&\cp1{2NT (w/stops)}&\\\hline
4-card unbid suit &\cg3{non-reverse}&\cv1{jump shift}\\
                &&&\cp2{reverse}\\\hline
6-card original suit &\cy2{rebid suit}&\cb1{jump rebid}\\\hline
\end{tabular}

\makeatletter
%\protected@write\@auxout{\string\def\expandafer\string\test{testing}}{
%\def\@@@{4}
%\protected@write\@auxout{}{\string\def\expandafter\string\csname test\@@@\endcsname{testing}}
\protected@write\@auxout{}{\string\gdef\expandafter\string\csname test\endcsname{testing}}
%\makeatother
% The point - use a counter to count the tables, then use the aux file
% to store the correct widths for all of them in a counted command.

\end{document}
