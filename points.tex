\documentclass[oneside]{memoir}
\usepackage{colortbl,footnote,fullpage}%,pointtables}
\usepackage{bridge}
\usepackage[perpage,hang,symbol*]{footmisc}
\usepackage{calc,graphicx}
\renewcommand{\hangfootparskip}{0pt}

\makeatletter

\newif\ifNoChapNumber
\makechapterstyle{VZ34}{
  \renewcommand\chapternamenum{}
  \renewcommand\printchaptername{}
  \renewcommand\printchapternum{}
  \renewcommand\chapnumfont{\Huge\bfseries}
  \renewcommand\chaptitlefont{\Huge\bfseries\raggedright}
%  \renewcommand\afterchapternum{\vspace*{-5pc}}
%  \renewcommand\insertchapterspace{\vspace*{-5pc}}
  \renewcommand\printchaptertitle[1]{%
    \begin{tabular}{@{}p{1cm}|!{\quad}p{\textwidth-1cm-2em-4\tabcolsep }}
      \ifNoChapNumber\relax\else\chapnumfont \thechapter\fi
      & \chaptitlefont ##1
    \end{tabular}
    \NoChapNumberfalse
  }
  \renewcommand\printchapternonum{\NoChapNumbertrue}
}
\makechapterstyle{AndersLyhne}{% doesn't seem to work?!
  \newlength{\chapterlineskipx}
  \setlength{\chapterlineskipx}{0.2cm}
  \setlength{\beforechapskip}{1.5cm}
  \setlength{\afterchapskip}{1cm}
  \setlength{\midchapskip}{2cm}
  \renewcommand\chapnamefont{\normalfont\normalsize\scshape\raggedleft}
  \renewcommand\chaptitlefont{\normalfont\normalsize\bfseries\sffamily\raggedleft}
  \renewcommand\chapternamenum{}
  \renewcommand\printchapternum{\makebox[0pt][1]{\hspace{0.2em}%
      \resizebox{!}{2ex}{\chapnamefont\bfseries\sffamily\thechapter}}}
  \renewcommand\afterchapternum{\par\hspace{1.5cm}\hrule\vspace{0.2cm}}
  \renewcommand\printchapternonum{\par}
  \renewcommand\printchaptertitle{\vskip\chapterlineskipx
    \hrule\vskip\afterchapskip}
}  

\chapterstyle{VZ34}
  

% We need to do stuff like this to pull things out of the aux file safely.
\ifx\test\undef
  \def\test{placeholder}
\fi

\let\xa\expandafter

\def\cellcol#1#2{\multicolumn{1}{>{\columncolor{#1}}c}{#2}}
\let\mc\multicolumn

\def\cellcol#1#2{\multicolumn{1}{>{\columncolor{#1}}c}{#2}}
\def\mc#1#2{\multicolumn{#1}{c}{#2}}
\def\mcc#1#2#3{\multicolumn{#1}{>{\columncolor{#2}}c}{#3}}

\definecolor{lblue}{rgb}{.75,.8,1}
\definecolor{lred}{rgb}{1,.8,.8}
\definecolor{lgreen}{rgb}{.8,1,.8}
\definecolor{lorange}{rgb}{1,.9,.7}
\definecolor{lpurple}{rgb}{.95,.7,1}
\definecolor{lyellow}{rgb}{1,1,.6}
\definecolor{laqua}{rgb}{.65,.95,1}

% See if we can't get better colors (both values and names and meanings)
% Forcing
% Game Forcing
% Invitational
% Slam asking?
% Asking/telling?
% Captain?
% - Striped background for responses which are off after interference
%   (various kinds for off after double and/or overcall??)

\newcommand{\cb}[2]{\mcc#1{lblue}{#2}}
\newcommand{\cp}[2]{\mcc#1{lred}{#2}}
\newcommand{\cg}[2]{\mcc#1{lgreen}{#2}}
\newcommand{\co}[2]{\mcc#1{lorange}{#2}}
\newcommand{\cv}[2]{\mcc#1{lpurple}{#2}}
\newcommand{\cy}[2]{\mcc#1{lyellow}{#2}}

\def\C{$\clubsuit$}
\def\D{$\diamondsuit$}
\def\H{$\heartsuit$}
\def\S{$\spadesuit$}
\def\NT{\textsc{nt}}
\let\TeXto\to
\let\TeXge\ge
\let\TeXle\le
\def\to{$\TeXto$}
\def\lt{$<$}
\def\gt{$>$}
\def\le{$\TeXle$}
\def\ge{$\TeXge$}
\def\ltsim{$\lesssim$}
\def\gtsim{$\gtrsim$}
\def\nodepth#1{\setbox\z@\hbox{#1}\dp\z@=0\p@\box\z@}
\def\killvert#1{\setbox\z@\hbox{#1}\dp\z@=0\p@\ht\z@=0\p@\box\z@}

% Defines command #1 which calls #2 with color in #3
\def\colorcommand#1#2#3{%
  \newcommand{#1}[2][*]{\ifx##1*\relax\def#3{{##2}}\else
    \def#3{[##1]{##2}}\fi#2}}

\newcommand{\@highlight}[2][2pt]{\begingroup
  % Make a colorbox to get padding
  \def\@cb{\expandafter\colorbox\@hl@col}%
  \def\x{\dimen\z@}\def\y{\dimen\@ne}\def\z{\dimen\tw@}\def\p{\dimen\thr@@}%
  \def\w{\wd\z@}\def\h{\ht\z@}\def\d{\dp\z@}%
  \setbox\z@\hbox{\@cb{}}%
  \p#1\relax\multiply\p\tw@ % twice the padding
  % \dimen0 is the highlighting width, \dimen1 is the height+depth
  \x-\w \y-\h \advance\y-\d
  \advance\x\p \advance\y\p
  % \dimen2 is just the depth
  \z-\d \advance\z#1\relax
  % Now set the text to get its dimensions
  \setbox\z@\hbox{#2}%
  \advance\x\w \advance\z\d
  \advance\y\h \advance\y\d
  % Set the actual colorbox
  \setbox\z@\hbox{\hskip-#1\@cb{\hbox to \x{\vbox to \y{}}}}%
  % Clear out its dimens and draw it (lowered)
  \w=0\p@\h=0\p@\d=0\p@\lower\z\box\z@\endgroup\hbox{#2}}
\colorcommand\highlight\@highlight\@hl@col

\makesavenoteenv{tabular}

\begin{document}

\def\singleton{1$^{\underline{\mathrm{tn}}}$}

\newif\ifcolorbids
\colorbidstrue

%\newcommand{\textbid}[2][white]{\ifcolorbids\colorbox{#1}{\killvert{\textbf{#2}}}\else\textbf{#2}\fi}
\newcommand{\textbid}[2][white]{\ifcolorbids\leavevmode\highlight{#1}[2pt]{\textbf{#2}}\else\textbf{#2}\fi}
\newcommand{\textresp}[2][white]{\emph{#2}}

% Dimensions: bw=remaining width; bh,bd=box height and depth
%             corr=correction for colorbox padding
\newdimen\scb@bw\newdimen\scb@bh\newdimen\scb@bd\newdimen\scb@corr
% This recursively typesets the stripes, one at a time
\def\scb@stripes{\let\scb@next\scb@stripes
  % First make sure we aren't over the bounds
  \ifdim\scb@bw<0\p@\let\scb@next\relax\else
  % See if we need to draw a partial stripe
  \ifdim\scb@bw<\scb@sw\relax
    % Correct the width if so, and make this the last loop
    \advance\scb@corr-\scb@sw\relax\advance\scb@corr\scb@bw\let\scb@next\relax
  \fi
  % Subtract off sw (stripe width) and ww (white width) from the remainder
  \advance\scb@bw-\scb@sw\global\advance\scb@bw-\scb@ww
  % Draw the colorbox and then skip over the white - could be generalized
  \expandafter\colorbox\scb@col{\lower\scb@bd\vbox to \scb@bh{}\hskip\scb@corr}\hskip\scb@ww\fi
  \scb@next}
% Typesets the two boxes, one on top of the other
\def\scb@scb#1{%
  % Calculate the width of the colorbox padding
  \setbox0\hbox{\expandafter\colorbox\scb@col{}}\scb@corr\scb@sw\advance\scb@corr-\wd0%
  % Put everything into an hbox, calculate the width of the text
  \hbox{\setbox0\hbox{#1}\scb@bw\wd0\scb@bh\ht0\advance\scb@bh\dp0\scb@bd\dp0%
    % Draw the stripes, then set it to zero size
    \setbox0\hbox{\scb@stripes}\wd0=0\p@\ht0=0\p@\dp0=0\p@
    % Typeset both boxes
    \box0\hbox{#1}}}
% Pull off the optional widths
\newcommand{\scb@getwidths}[1][5pt]{\scb@parsewidths#1,#1,\@@\scb@scb}
% Parse the widths and stick into variables
\def\scb@parsewidths#1,#2,#3\@@{\def\scb@sw{#1}\def\scb@ww{#2}}
% Grab the color into col, then keep processing
\newcommand{\stripedcolorbox}[2][*]{\ifx#1*\relax\def\scb@col{{#2}}\else
  \def\scb@col{[#1]{#2}}\fi\scb@getwidths}

\iffalse
\test
\newcolumntype{C}{>{\phantom{m}}c<{\phantom{m}}}
\newcolumntype{W}{c<{\phantom{m}}>{\phantom{m}}c}
\newcolumntype{R}{c<{\phantom{m}}>{\phantom{n}}c<{\phantom{n}}>{\phantom{m}}c}

\begin{tabular}{lcccccccccc}
\hline
\cellcol{black}{} & \co1{0--6} & \cb1{6--8} & \cg2{8--10} & \cp1{10--12} & \cv1{13--14} & \cp1{15--16} & \cg1{17--19} & \cb1{20--22} & \co1{23+}\\\hline\hline
Opening && \co4{Weak 2} & \cb2{Min} & \cg1{Inv} & \cp1{Max} & \cv1{Strong 2}\\\hline
Response && \cb3{Weak} & \cg1{Inv} & \cp5{Strong}\\\hline\hline

Direct o/c &&&\phantom{r}& \cg7{Overcall}\\\hline
Takeout Dbl &&&&&&&& \cp3{Strong}\\\hline

Invit'l cuebid &&&&\phantom{r}& \cb1{Raise} & \cp5{Raise or Shift}\\\hline
Michaels && \co4{Weak} &&& \cp3{Strong}\\\hline
Unusual NT && \co4{Weak} &&& \cp3{Strong}\\\hline
\hline
\end{tabular}

\colorbox{lblue}{\vbox to 2 in{\hbox to 2in{Testing 12 13 14 15}\hbox to 2in{a b c d ef\hss}}}
% Could make i.e. \row[height=2pc]\cell[4]{...}\cell[3]{...}
% in terms of units set at the top..  But how to scale properly?
% Would like to make it as narrow as possible, to accomodate everything.
% We could accumulate box widths and then divide to keep a running total
% of the minimum length of the unit.

%\beginbidtable abc\endbidtable
\eject
\fi

%\title{Bridge Bidding}
%\author{Arranged by Stephen Hicks}

%\maketitle

\section*{Bridge Bidding\hfill Arranged by Stephen Hicks}

%{\textbf{Bridge Bidding}\hfill Arranged by Stephen Hicks}
%\vskip 2pc

%\chapter{Openings}

%\begin{abstract}

% Maybe move this to a page with the conventions on it, and expand
% the ToC to full page

\subsection*{Openings}
\begin{ptable}\headings{0-5|6-8|9-12|13|14|15|16|17|18|19|20|21|22|23+}
\row{balanced}
%  \[13-14,G]{...}
  \bid(13-14,lgreen){}
%  \bid(15-17,lblue){\vbox{\hbox{\vsize=.5pc\tiny 1NT}\hbox{\vsize=.5pc\colorbox{lgreen}{\tiny1/1,2/1}}}} % UGLY...
  \bid(15-17,lyellow){1NT}
  \bid(18-19,lgreen){}
  \bid(20-22,lred){2NT}
  \bid*(23,lpurple){2\C}\\%\hline
\row{openable suit\footnote{Opening a suit requires 5 of a major or 4 of a minor}}
  \bid*(13-22,lgreen){1\C\D\H\S}
  \bid*(23,lpurple){2\C}\\%\hline
\row{6 card suit}
  \bid(6-12,lblue){2\D\H\S}
  \bid(13-22,lgreen){1\C\D\H\S}
  \bid(23,lpurple){\clift{2\C\w{a}}}\\%\hline
\row{7 card suit}
  \bid(0-8,lorange){\ge3\C\D\H\S}
  \bid*(9-12,lblue){2\D\H\S}
  \bid*(13-22,lgreen){1\C\D\H\S}
  \bid*(23,lpurple){2\C}\\
\spanrow{5}{solid minor, 2 outside stoppers\footnote{7-card minor (sometimes 6), usually all 3 outside suits stopped, never a singleton or void}}
  \bid(16-21,laqua){Acol 3NT\w{a}}\\
\end{ptable}

\subsection*{3rd and 4th seat}
Open one of a suit with fewer points.  Bid weak two with shorter suit
or more points.
%\end{abstract}

\vspace*{-1.5pc}
\tableofcontents*

%% Hack of colortbl:
%\def\@lf{\hbox{\vrule\hbox to 1pt{}}}
%\def\fake#1{\setbox0\hbox{#1}\wd0=0.5pt\ht0=0pt\dp0=0pt\box0}
%\def\fakes#1{\let\@next\relax\ifnum\count@>\z@\fake#1\advance\count@\m@ne\let\@next\fakes\fi\@next}
%\def\CT@@do@color{%
%  \global\let\CT@do@color\relax
%        \@tempdima\wd\z@
%        \advance\@tempdima\@tempdimb
%        \advance\@tempdima\@tempdimc
%        \kern-\@tempdimb
%        \leaders\hbox{\large\fake/\fake/\fake/\fake/\fake/\fake/\hskip 3.1pt}%\vrule
%%^^A                     \@height\p@\@depth\p@
%                \hskip\@tempdima\@plus  1fill
%        \kern-\@tempdimc
%        \hskip-\wd\z@ \@plus -1fill }

\chapter{Suit openings}

\section{Responses to 1 of a suit}
%\textbf{Limit Major Raises}, \textbf{Jacoby 2NT},
%\textbf{Splinter Bids}, \textbf{Weak Jump Shifts}, and \textbf{Inverted
%Minor Raises}\\

%\begin{tabular}{cc}
%  abcdefg&hijklmnop\\
%  {\color{green}\leaders\hbox{/\hskip-3pt}\hfill}\hskip 0pt\@plus-1fill test\\
%\end{tabular}

\begin{ptable}\headings{2-6|6-10|11-12|13+}
\row{three-card major suit fit}
  \bid*(6-10,lblue){single raise (2M)}
  \bid*(11-12,lgreen){limit raise (3M)}
  \bid(13,laqua){1/1 or 2/1}\\
\row{four-card major suit fit}
  \bid(6-10,lblue){single raise (2M)}
  \bid(11-12,lgreen){limit raise (3M)}
%  \bid(6-12,white){\ditto{2in}}
  \bid(13,lpurple){Jacoby 2NT\w{ago}}\\
%\row{\ldots\ + 1$^{\underline{\mathrm{tn}}}$/void}
\row{\ldots\ + singleton/void}
  \bid*(6-10,lblue){single raise (2M)}
  \bid*(11-12,lgreen){limit raise (3M)}
%  \bid(6-12,white){\ditto{2in}}
  \bid(13,lyellow){splinter bid\footnote{Splinter bid is a double-jump (non-game) bid in new suit, showing a singleton.}\w{ag}}\\\\[-.6pc]
%\row{5 trump fit}
%  \bid(7-10){preempt to game (4M)}
\row{four-card minor suit fit}
  \bid(6-10,lblue){jump raise (3m)\footnote{Prefer 1NT with a bare 10-point raise}}
  \bid(11-13,lgreen){single raise (2m)}\\\\[-.6pc]
\row{\drop{new suit\footnote{All new suits require at least 4 cards.
After a minor suit opening, prefer bidding 4-card suits up the line
at the 1-level.  2\H\ requires a five-card suit.  Weak jump preempts
require roughly 6 cards at the 2-level and 7 at the 3-level.}}}
  \bid*(2-6,lorange){jump shift}
  \bid(6-10,lblue){1NT\footnote{\label{ntresp}1NT response/rebid denies a biddable 1-level major}}
  \bid(11-13,lgreen){2/1}\\
\row{}%\row{\phantom{m}\ditto{0.6in}}
  \bid(2-6,lorange){\clift{jump shift}}
  \bid(6-13,laqua){1/1}\\%\\[-.6pc]
%\row{six-card suit}
%  \bid(2-6,lorange){jump shift\savenote{\weakshift}{Weak jump shifts 
%      at the three level \emph{must} have a slightly longer suit, 
%      and \emph{may} show an extra point or two}}
%  \bid(6-13,white){\ditto{3in}}\\
%four-card suit%\footnote{2\H requires 5 cards}
%(5 to bid 2\H)&\co1{preempt\footnote{double jump shift requires 7-card suit}}&\cy1{1NT}&\cg4{2/1}\\
\end{ptable}

\section{Major suit fits}


\section{Minor suit fits}


\section{New suits}


\chapter{Notrump openings}

Promise balanced hand (at most 1 doubleton).\\
\begin{ptable}\headings{13-14|15-17|18-19|20-22|23+}
\row{}
  \bid(13-14,lblue){open suit, rebid NT}
  \bid(15-17,lyellow){1NT}
  \bid(18-19,lgreen){open suit, jump rebid NT}
  \bid(20-22,lred){2NT}
  \bid(23,lpurple){2\C, rebid NT}\\
\end{ptable}

\section{Responses to 1NT}
\begin{ptable}\headings{0-8|8-10|11-12|13-15|16-17|18+}
\row{(semi)balanced}
  \bid(8-10,lblue){2NT}
  \bid(11-15,lred){3NT}
  \bid(16-17,lpurple){4NT}
  \bid(18,lyellow){6NT}\\
\row{five-card major}
\iffalse % Texas Transfers
  \bid(0-10,lgreen){Jacoby transfer 2\D\H}
  \bid(11-15,lblue){Texas transfer 4\D\H}
  \bid(16-18,lgreen){2\D\H}\\
\else
  \bid(0-18,lgreen){Jacoby transfer 2\D\H}\\
\fi
\row{four-card major}
  \bid(0-8,lblue){2\C--Pass\footnote{only pass Stayman response with 4441 shape, 0--8 points}}
  \bid*(8-18,lyellow){Stayman 2\C}\\
\row{5 or 6-card minor}
  \bid(0-8,lorange){3\C\D\footnote{minor suit bust requires 6 cards}}
  \bid(11-18,lyellow){\clift{Stayman 2\C}}\\
\end{ptable}
%\hskip 1pc
%\parbox[t]{1.8in}{
%  2\C-?-3m = no 4-card major}

\section{Responses to 2NT}
\begin{ptable}\headings{0-3|4|5-7|8-10|11+}
\row{(semi)balanced}
  \bid(5-10,lblue){3NT}
  \bid(11,lred){4NT}\\
\row{five-card major}
  \bid(0-11,lgreen){Jacoby transfer 3\D\H}\\
\row{four-card major}
  \bid(4-11,lyellow){Stayman 3\C}\\
\row{5 or 6-card minor}
%  \bid(0-7,lorange){3\C\D (w/ 6+ cards)}
  \bid*(8-11,lyellow){Stayman 3\C}\\
\end{ptable}

\section{Stayman}

\section{Jacoby}


\chapter{Artificial openings}

\section{Strong 2\C}

Opener should open 2\C\ with 23 high card points and a balanced hand.
With an unbalanced hand, 21 high card points is sufficient.  With a
one-suited hand, opener must have within one trick of game and at least
17 high card points.  This bid is artificial and absolutely forcing,
asking for at least two responses from partner (unless bidding
reaches game).

\subsection{Responses}
\begin{ptable}\headings{0-7|8+}
\row{good 5-card suit}\bid*(0-7,lorange){2\D\w{a}}\bid(8,lpurple){2M, 3m}\\
\row{balanced}\bid(0-7,lorange){\clift{2\D}}\bid(8,lred){2NT}\\
\end{ptable}
\hskip 1pc\vrule\hskip 1pc
\parbox[t]{4in}{\vskip -1.4pc %Solid suit headed by two of top three honors:\\
%\phantom.\hskip 1pc 6-card major: \colorbox{lblue}{3M}\hskip 1.9pc
%7-card minor: \colorbox{lgreen}{4m}\\
%Undisclosed solid suit headed by A-K-Q: \colorbox{lyellow}{3NT}
\textbf{Special responses}: with a 6+ card major or 7+ card minor headed by two
of the top three honors, bid \textbid[lblue]{3M} or
\textbid[lgreen]{4m}.  With an undisclosed solid suit headed by A-K-Q,
bid \textbid[lyellow]{3NT}.  }

\subsection{Opener's rebids (all forcing)}
\begin{ptable}\headings{23-24|25-27|28-30|31-32|33+}
\row{5-card suit\footnote{strong 4-card suit if 4441}}\bid(23-33,lyellow){2M, 3m}\\
\row{balanced}\bid(23-24,lblue){2NT}\bid(25-27,laqua){3NT}
\bid(28-30,lgreen){4NT}\bid(31-32,lred){5NT}\bid(33,lpurple){6NT}\\
\end{ptable}
\hskip .5pc\vrule\hskip .5pc
\parbox[t]{2.9in}{\vskip -2pc
\textbf{After a positive} 2NT response, \textbid[lyellow]{Stayman 3\C} is on.
After a positive suit response, \textbid[lblue]{shift} with a new 5-card
suit, \textbid[lgreen]{raise} with support, or bid \textbid[laqua]{NT}
with a balanced hand.}
%\parbox[t]{3in}{\vskip -2pc
%After 2NT resp. w/ 4-card major: \colorbox{lyellow}{Stayman 3\C}\\[0.6pc]
%After positive suit response:\\
%\phantom.\hskip 1pc
%  5 cards:\colorbox{lblue}{shift}
%  support:\colorbox{lgreen}{raise}
%  balanced:\colorbox{lred}{NT}
%}

\subsection{Responder's rebids}
\begin{ptable}\headings{0-3|4+}
\row{after first negative}\bid(0-3,lorange){cheaper minor\w{a} or 3NT\w{a} after 3\D}
\bid(4,lred){natural}\\
\end{ptable}

\subsection{Opener's rebids after second negative}
Opener may \textbid[lorange]{raise first suit\ww{s}} as only non-game-forcing
rebid.  All else is natural and game-forcing.

\subsection{After interference}
\iffalse % Table format
\begin{ptable}\headings{0-4|5-7|8+}
\multicolumn4c{After double (other responses on)}\\
\row{4 clubs}\bid(0-8,lblue){pass}\\
\row{5 clubs (4 good clubs)}\bid(0-8,lyellow){redouble}\\
\multicolumn4c{After overcall beyond 4\D}\\
\row{good 6-card or longer suit}\bid(0-7,lorange){double}
  \bid(8,lyellow){bid suit}\\
\row{want to hear opener's suit}\bid*(0-4,lorange){double}
  \bid(5-8,lred){pass}\\
\end{ptable}\hfill
\begin{ptable}\headings{0-4|5-7|8+}
\multicolumn4c{After overcall through 4\D}\\
\row{5-card suit}\bid*(0-4,lorange){pass}\bid(5-8,lgreen){bid suit}\\
\row{long enemy suit}\bid*(0-4,lorange){pass}\bid(5-8,laqua){penalty double}\\
\row{balanced + 1 stopper}\bid*(0-7,lorange){pass}\bid(8,lred){NT}\\
\row{\singleton/void in enemy suit}\bid(0-7,lorange){\clift{pass}}
  \bid(8,lpurple){cue-bid}\\
\end{ptable}
\fi

% Text format
\noindent \textbf{If 2\C is \emph{doubled}}, \textbid[lblue]{pass}
with 4 clubs, \textbid[lyellow]{redouble} with 5 clubs (or 4 good
clubs), \textbid[laqua]{normal responses} on.
\vskip0.25pc\hrule\vskip0.25pc
\noindent \textbf{If 2\C is \emph{overcalled through 4\D}}, responder may bid a
\textbid[laqua]{five-card suit} with 5+ HCP, bid \textbid[lgreen]{NT}
with a balanced 8 points and a stopper, \textbid[lred]{cue-bid\ww{a}}
with 8 points and a singleton/void in enemy suit, \textbid[lyellow]{double}
for penalty, or \textbid[lorange]{pass} otherwise (negative response).
\vskip0.25pc\hrule\vskip0.25pc
\noindent \textbf{If 2\C is \emph{overvalled beyond 4\D}}, responder may
\textbid[lorange]{double\ww{a}} with a \emph{minimum} hand to warn against any
rebidding, \textbid[lpurple]{pass\ww{a}} to encourage opener to bid his suit,
or \textbid[lblue]{bid a good 6-card suit}.
\vskip0.25pc\hrule\vskip0.25pc
\noindent \textbf{For interference after response},
\textbid[laqua]{normal rebids} are on, \textbid[lyellow]{double} or
\textbid[lyellow]{redouble} for penalty, \textbid[lred]{cue-bid} to show
other three suits, or \textbid[lpurple]{pass\ww{f}} to force responder
to act.

\section{Weak two-bids}

2\D, 2\H, and 2\S\ are weak preemptive bids requiring at least a six-card
suit and 5--12 high card points.  Responder should generally pass unless
he has something constructive to add, and should absolutely not attempt
to rescue.  Another suit should only be bid as a better
preempt with no help from opener.

\subsection{Responses}
Any \textbid[lblue]{raise} (2\D---3\D) or
\textbid[laqua]{game bid} (2\C---3NT, 2\H---4\S) is a further preempt
and \textresp{must be passed}.  A \textbid[lyellow]{nonjump shift}
(2\C---2\S, 2\H---3\D) shows 6 cards, no fit, and presumes a better
preempt.  A \textbid[lgreen]{jump shift} (2\D---3\H) is natural and
invitational, showing an independent suit one trick short of actual
bid.  After either shift, opener should \textresp{usually pass} but
\textresp{may raise}.  \textbid[lred]{2NT\ww{a}} is the only forcing
response, showing 14+ points.

\subsection{Weak 2-bidder's rebids after a 2NT response}
\begin{ptable}\headings{5-8|8-12}
  \row{A-K-J or better in long suit}
    \bid*(5-8,lorange){rebid original suit}
    \bid(8-12,lblue){3NT\w{a}}\\
  \row{A, K, or Q in side suit}
    \bid(5-8,lorange){rebid original suit}
    \bid(8-12,lyellow){3 of side suit\w{f}}\\
  \row{5-card\footnote{or a good 4-card minor headed by at least Q} minor side suit}
    \bid*(5-8,lorange){rebid original suit}
    \bid(8-12,lgreen){4m\w{f}}\\
\end{ptable}

\subsection{2NT responder's rebids}
Responder should know enough to place final contract.  Responder may
\textbid[lorange]{pass} a weak rebid, or 3NT with stoppers.  A
\textbid[lblue]{rebid of the opened suit} is a sign-off, but mildly
invitational if below game.  Any \textbid[laqua]{game bid\ww{s}} is a
natural sign off.  Any other \textbid[lgreen]{new suit bid\ww{f}} shows
a 5-card suit and is forcing (opener should \textresp{raise} with three
card support or a doubleton honor, or otherwise \textresp{return} to
opened suit).

\subsection{Interference}
\textbf{If weak two-bid is \emph{doubled}}, \textbid[lred]{redouble} shows a
defensively oriented hand with at least 14 high card points.\\
\textbf{If weak \emph{two-bid is overcalled}}, normal responses are on,
\textbid[lyellow]{double} is for penalty, or \textbid[lred]{cue-bid}
for an all-purpose slam try (except \D, which may be an attempt at
3NT, asking for enemy stopper).\\
\textbf{If \emph{2NT response is overcalled}},
opener may \textbid[lorange]{pass} with minimum, \textbid[lblue]{double}
to show feature (A, K, or Q) in enemy suit, bid \textbid[laqua]{3 of original
suit} to show feature in an unavailable suit, or make any other normal rebid.

% Organization: Do we want to do all the responses first,
% and then the rebids, or do we want to separate out the
% various types of auctions and show all responses and rebids
% from there?  Probably the latter...

\eject

\section{Acol 3NT}
Acol 3NT opening is a gambling 3NT, promising 16--21 points, a 7-card
(or good 6-card) minor, at least two (usually 3) outside stoppers, and
no singleton or void.  Responder must never attempt to rescue and only
bid with a better game or to attempt to find a slam.

\subsection{Responses}
\textbid[lred]{4\C} is a slam try, promising 9 points, including an
ace and king, or three kings (opener may bid \textresp{4NT} or
\textresp{5m} to deny slam try, or \textresp{cue-bid} a major suit
ace, or bid \textresp{4\D} Gerber).  \textbid[lpurple]{4\D\ww{a}} at
any time by either player is Gerber, asking for aces.
\textbid[lgreen]{4M} is nonforcing, showing a good 6-card or longer
suit, denying an outside ace.  \textbid[lyellow]{4NT} is natural slam
invite, showing 11--13 points but insufficient controls for 4\C.
\textbid[lblue]{5\C} or \textbid[lblue]{6\C} at any time (including
after asking Gerber) by responder requests sign-off in opener's minor
suit (pass or correct).

\subsection{Interference}
\textbf{If 3NT is \emph{doubled}}, \textbid[lorange]{4\C} is a rescue bid,
\textbid[lblue]{pass} with a few scattered values, \textbid[lred]{redouble}
to show normal 4\C response, or any \textbid[laqua]{normal response}.

\noindent\textbf{If 3NT is \emph{overcalled}}, \textbid[lgreen]{4NT}
is competitive, \textbid[lyellow]{double} is for penalty,
\textbid[laqua]{normal responses} are still on (including
\textbid[lorange]{pass} with minimum).

\noindent\textbf{If \emph{responder passes} overcall}, \textbid[lred]{bid}
or \textbid[lred]{double} with maximum, \textbid[lorange]{pass} with minimum.

\chapter{Defensive bidding}
\begin{ptable}\headings{0-8|9-10|11-12|13-14|15-17|18+}
\row{5-card suit}
\bid(9-17,lblue){nonjump overcall}
\bid(18-,lpurple){double}\\
\row{all unbid suits}
\bid(9-,laqua){double\footnote{lower point limit depends on quality of 
shape (lower with eg 5440), major honors, and vulnerability, and may pass 
with high as 13}}\\
\row{\drop[1pc]{two-suited\brace{2.5em}{1.6pc}}}
\bid(15-,lgreen){double w/ 5-card minor and 4-card major}\\
\row{}
\bid(9-12,lyellow){Michaels}
\bid(18-,lred){Michaels}\\
\row{}
\bid(9-12,lyellow){Unusual 2NT}
\bid(18,lred){Unusual 2NT}\\
\row{balanced}
\bid(15-17,lgreen){1NT}\\
\row{independent suit}
\bid(0-10,lorange){jump overall}\\
\end{ptable}\\
With 21 or more points and any shape, double then cue-bid.

\section{Overcalls}
\subsection{After partner's direct (nonjump) overcall}
\begin{ptable}\headings{5|6-7|8|9|10|11|12|13-14|15+}
\row{3+ trump fit}
  \bid*(6-10,lblue){single raise}\bid*(11-,lyellow){cue-bid}\\
\row{4+ trump + singleton/void}
  \bid(6-10,lblue){\clift{single raise}}
  \bid(11-12,lyellow){\clift{cue-bid}}
  \bid(13-,lpurple){jump cue-bid}\\
\row{new 5-card suit}
  \bid(8-12,laqua){nonjump shift\footnote{need longer suit for higher level}}
  \bid*(13-,lyellow){cue-bid}\\
\row{new 6-card suit}
  \bid(11-12,lgreen){jump shift}
  \bid(13-,lyellow){\clift{cue-bid}}\\
\row{stopper in enemy suit\footnote{suggests (semi)balanced and stoppers in
    all unbid suits; points given
    for 1-level overcall---reduce by 2--3 after 2-level overcall}}
  \bid(8-11,lblue){1NT}
  \bid(12-14,laqua){2NT}
  \bid(15-,lred){3NT}\\
\row{4+ trump fit}\bid(5-8,lorange){jump raise}\\
\end{ptable}

\subsection{After two-suited overcall}

\section{Takeout doubles}
Asks partner to bid a (typically) 4-card suit, with preference for
majors.  After RHO opens 1 of a suit, it means opening strength with
support (3-card) for the other three suits, or a strong hand (17+) with any
distribution.\footnote{Points may be shaded slightly with 4-card support
for all other suits.  Meaning is same after 1NT response to LHO's opening
suit.  After two suits, requires 4-4 in unbids (fewer points with more
distribution).  After a strong action, shows freak distribution and a
few honors.  After a previous double, shows strong hand (17+).  After
previous overcall of opponents, shows more strength (15+) and at least
3-3 in other suits.  After a free pass (no previous bids), shows
maximal passed hand and support for unbid suits.  After a non-free
pass, if new information has come since the passed-up chance to
double, then shows support for the two unbid suits
(1\S-P-1NT-P-2\C-X); otherwise, with no new information, is for
penalty (1\S-P-1NT-P-2\S-X).}

\textbf{If partner has not bid or doubled,
the double of a \emph{natural suit} bid through \emph{4\D}} is for takeout.

\subsection{Responses}
It is important to show strength with the response, since doubler is
captain.

\begin{ptable}\headings{0-5|6-9|10|11-12|13-16|17+}
\row{6-card suit}
  \bid(0-9,laqua){double jump}
  \bid*(10-12,lgreen){jump}
  \bid*(13-,lpurple){cue-bid}\\
\row{longest suit}
  \bid*(0-9,lorange){minimum}
  \bid(10-12,lgreen){\clift{jump}}
  \bid(13-,lpurple){cue-bid}\\
\row{4-4 majors}
  \bid(0-9,lorange){\clift{minimum}}
  \bid*(10-,lpurple){cue-bid}\\
\row{enemy stopper\footnote{denies 4-card unbid major}}
  \bid*(0-5,lorange){minimum}
  \bid(6-9,lblue){1NT}
  \bid(10,lyellow){\hskip-7pt\colorbox{lblue}{\hskip 10pt}}
  \bid(11-12,lyellow){2NT}
  \bid(13-16,lred){3NT\footnote{slightly less with a long minor for extra tricks, but must have minimal doubt}}
  \bid*(17-,lpurple){cue-bid}\\
\end{ptable}\\
A takeout double may only be passed with five trump, from which
three must be winners.

\subsubsection{Responses in Competition}
After a redouble, pass shows no clear preference between unbid suits.
After an overcall, pass shows 0-5 points, double is for penalty (keeping
in mind partner's promised length).

\subsection{Rebids by doubler after minimal response}

\begin{ptable}\headings{9-15|16-17|18|19-20|21-22|23+}
\row{enemy stopper}
  \bid*(9-17,lorange){pass}
  \bid(18-20,lblue){1NT}
  \bid(21-22,lyellow){2NT\footnote{nonjump 2NT with 19--21}}
  \bid(23-,lred){3NT\footnote{must expect to take 9 tricks}}\\
\row{5-card new suit}
  \bid(9-17,lorange){\clift{pass}}
  \bid(18-20,lgreen){new suit}
  \bid*(21-,lpurple){cue-bid}\\
\row{4-card support}
  \bid*(9-15,lorange){pass}
  \bid(16-18,lblue){single raise\footnote{single raise to 3 requires 18--20}}
  \bid(19-20,laqua){jump to 3}
  \bid(21-,lpurple){\clift{cue-bid\footnote{responder should jump bidding with any values}}}\\
\end{ptable}

\subsection{Rebids by doubler after non-minimal response}

After jumped response (game invite), natural bid accepts, pass rejects.
After cue-bid (game-forcing), all bids natural.  After 1NT (6--10 points),
2NT invites, 3NT signs off, 5-card suit nonforcing at 2-level, forcing at
3-level, cue-bid invites game and asks for a suit, pass is to play.

\chapter{Offensive bidding}


\section{Negative doubles}

\section{Overcalls}


\chapter{Slam bidding}

\section{Blackwood and Gerber}

\section{Cue-bidding}


\chapter*{Extra crap}


\section*{Opener's rebids after 1/1}
\begin{tabular}{lcccc}
&\mc2{13--16}&\mc1{17--19}&\mc1{20--22}\\\hline\hline
fit (4 trump)&\cb2{single raise}&\cg1{jump raise}&\cv1{reverse, jump shift}\\\hline
balanced & \cy1{1NT\footref{ntresp}%{1NT response after 1/1 gaurantees no 4-card major at 1-level}%
}&&\cp1{2NT (w/stops)}&\\\hline
4-card unbid suit &\cg3{non-reverse}&\cv1{jump shift}\\
                &&&\cp2{reverse}\\\hline
6-card original suit &\cy2{rebid suit}&\cb1{jump rebid}\\\hline
\end{tabular}

\subsection{1/1 bidder's rebids after opener's single raise}
6--10 points: Pass\\
11--12 points: Invite game in suit only at 3-level.  Invite game in either
suit or notrump with 2NT.  An unbid suit invites game in agreed-upon suit,
but shows side honors or length.  3 of opener's original suit shows only
4-cards in own suit, but good fit in opener's suit, invites game.\\
13+ points: Bid game, or bid or investigate slam

\subsection{1/1 bidder's rebids after opener rebids 1NT (minimum)}
6--10 points: Scramble for a safe part-score (non-reverse, non-jump)\\
11--12 points: 2NT, or over/under-bid\\
13+ points: Reverse or jump bidding

\subsection{1/1 bidder's rebids after opener's nonjump rebid of original suit}
6--10 points: Pass, unless \le1 trump \emph{and} a 6-card suit to escape to\\
11--12 points: Show another (4-card) suit (forcing one round), or 2NT with
unbid suits stopped, or raise with as meager as 2-card major or 3-card minor 
support\\
13+ points: Bid game if known, or forcing new suit otherwise

\subsection{1/1 bidder's rebids after opener's non-reversing shift to unbid suit}
6 points: Pass\\
7--10 points: 1NT if balanced and no fit, or show preference for one of opener's suits at 2-level\\
11--12 points: (non-jumping) 2NT with last suit stopped, fourth suit (forcing),
invite game in one of opener's suits with non-jump to 3-level; jump from a 
passed hand\\
13+ points: jump
\subsubsection{1NT responder's rebids after opener's non-reversing shift}
Prefer second suit with 4-card support, bid 6-card suit at 2-level, or 7-card
suit at 3-level.

\subsection{1/1 bidder's rebids after opener's reverse}
6--8 points: 2NT with last stopper, or nonjump in either of first two suits\\
9+ points (game-forcing): fourth suit, raise third suit (4 trump), jump first suit, 3NT

\subsection{1/1 bidder's rebids after opener's jump bid}
\subsubsection{After opener's jump raise, jump rebid, or 2NT}
Pass with 6--7, bid suit or NT game with 8+.
\subsubsection{After opener's jump shift}
Bid naturally to show strength.  Game is required.

\section*{Opener's rebids after 2/1}
% GRR...  these tables are a pain, especially for stupid systems
\begin{tabular}{lcccc}
&\mc1{13--14}&\mc1{15--16}&\mc1{17--19}&\mc1{20+}\\\hline\hline
fit (4 trump)&\cb2{single raise}&\cg1{jump raise}&\cv1{reverse, jump shift}\\\hline
balanced & \cy1{1NT}&&\cp1{2NT (w/stops)}&\\\hline
4-card unbid suit &\cg3{non-reverse}&\cv1{jump shift}\\
                &&&\cp2{reverse}\\\hline
6-card original suit &\cy2{rebid suit}&\cb1{jump rebid}\\\hline
\end{tabular}

\makeatletter
%\protected@write\@auxout{\string\def\expandafer\string\test{testing}}{
%\def\@@@{4}
%\protected@write\@auxout{}{\string\def\expandafter\string\csname test\@@@\endcsname{testing}}
\protected@write\@auxout{}{\string\gdef\expandafter\string\csname test\endcsname{testing}}
%\makeatother
% The point - use a counter to count the tables, then use the aux file
% to store the correct widths for all of them in a counted command.

\end{document}
