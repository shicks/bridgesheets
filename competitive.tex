\documentclass[10pt]{article}
\usepackage{amssymb}

%\advance\textheight by 1pc

\title{Defensive Bidding}
\author{Stephen Hicks}

\def\C{$\clubsuit$}
\def\D{$\diamondsuit$}
\def\H{$\heartsuit$}
\def\S{$\spadesuit$}
\def\NT{\textsc{nt}}
\let\TeXto\to
\let\TeXge\ge
\let\TeXle\le
\def\to{$\TeXto$}
\def\lt{$<$}
\def\gt{$>$}
\def\le{$\TeXle$}
\def\ge{$\TeXge$}
\def\ltsim{$\lesssim$}
\def\gtsim{$\gtrsim$}

%\font\eightrm=cmr8
\font\exceptparfont=cmr10% at 9pt
\font\exceptfont=cmr10 at 8.2pt
%\def\small{\ninerm}

\def\except#1{\paragraph{\exceptparfont\bf Exceptions:}{\exceptfont #1}}
\renewcommand{\bullet}[1]{\begin{itemize}\item#1\end{itemize}\vspace*{-0.5pc}}
\newcommand{\crunch}[1][1]{\vspace*{-#1pc}}
\newenvironment{mylist}[1][.5]{\begin{itemize}\itemsep=-#1\baselineskip}{\end{itemize}}

\begin{document}
\maketitle%\vspace*{-2pc}
%\vspace*{-3pc}
%{\hfill{\huge Competitive Auctions}\hfill}
%\vskip 6pt{\hfill{\Large Stephen Hicks}\hfill}
%\vskip 6pt{\hfill{\large\today}\hfill}


\section{Overcalls}
Reasons to overcall: make a contract, a sacrifice, push opponents
too high, suggest an opening lead, hassle opponents.  But only overcall
when hand qualifies.  Qualifications are very broad, but depend heavily
on factors such as level of bidding and vulnerability.

\subsection{Direct overcalls}
When your right-hand opponent (RHO) opens the bidding, one way to enter
the bidding is by making an overcall.  Such an overcall is known as 
\emph{direct}.  Situations also arise when your partner has passed and
both opponents have bid once.  None of the overcalls are forcing.

\subsubsection{... at the one-level}
\bullet{9--17 points, good 5-card suit}
With 18+ points, a takeout double is more appropriate.\crunch
\except{A worthless 5-card suit or an amazing 4-card suit is OK with an 
otherwise strong hand.  With as many as 19 points and a two-suited (5-5 or
more) hand, a suit overcall is more appropriate due to bad rebids after
a takeout double.  After a 1\NT\ opening, a much stronger hand is required
(13--17 points and an amazing suit or exciting distribution).}

\subsubsection{... at the two-level}
\bullet{9--17 points, good 5-card suit}
A better hand is required for two-level overcalls.  With minimal points,
a substantial (6-cards and honors) suit is required, especially with 
unfavorable vulnerability.\crunch
\except{Beware of overcalling with a doubtful hand and moderate length in
enemy suit.  At the three level, a few more points \emph{and} a better and
longer suit is required.}


\subsection{Invitational cue-bid response to direct overcalls}
\begin{mylist}[.3]
\item 11+ points and support for overcaller's suit, or 13+ points in new suit.
\item Weaker hands must use other responses.
\item[!!] Jump cue-bid resp. is mini-splinter: 13+ points, 4 trump fit, singl./void.
\end{mylist}

\subsubsection{Rebids by overcaller}
Rebids should all be natural, subject to the following constraints:
\begin{mylist}[.3]
\item Pass only if RHO overcalls cue-bid and nothing constructive to say.
\item Rebid original suit w/ minimum o/c, not forcing (don't need \gt5 cards).
\item Rebid \textbf{below} original suit w/ minimum or maximum o/c, forcing.
\item Rebid \textbf{above} orig. suit w/ good or maximum o/c, forcing if below game.
\end{mylist}
\subsubsection{Rebids by cue-bidder}
\begin{mylist}[.3]
\item Rebid overcaller's first suit w/ 11--12 points, not forcing.
\item Raise overcaller's second suit (below game) w/ 13--14, invites game.
\item New suit w/ 13+ points, 5-card suit, forcing one round.
\item 2\NT\ shows 13--14 points after 1-level o/c, 11--12 after 2-level, invites game.
\item[!] Repeat original cue-bid to show a natural suit, invitational if below game.
\item Game bid is natural sign-off attempt.
\end{mylist}

\subsubsection{In competition}
If RHO bids after partner overcalls or doubles, invitational cue-bid is still 
on and typically shows fit in partner's suit and 11--12 points.  If RHO
bids a new suit, cue-bid \emph{stronger} of the two choices (showing stoppers
and helping partner reevaluate his honors).

\subsubsection{From a passed hand}
Shows 11--12 points and fit.  May be otherwise if pass was after opponent
opened bidding.  If opponents have shown two suits, second is invitational
cue-bid and first suit is natural long suit.

\subsubsection{Other applications}
Invitational cue-bids are on after weak jump raises, Michaels cuebids,
and unusual \NT.

\subsection{Limited responses to direct overcalls}
After partner overcalls and next opponent passes, you have the option to
respond.  The only strong/forcing response is the invitational cue-bid.
Every other response  (including jumps) is weak and non-forcing.

\subsubsection{Raising partner's suit}
\begin{mylist}[.3]
\item Single raise: 6--10 points, 3 trump fit.
\item Jump raise (any level): preemptive, 5--8 points if below game, 4+ trump.
\end{mylist}%\crunch[1.5]

\subsubsection{Responding in a new suit}
\begin{mylist}[.3]
\item Nonjump: 8--13 points.  (@1---5 cards; @2---strong 5; @3---6 cards)
\item Jump: 11--13 points, strong 6-card or longer suit, invites game.
\end{mylist}

\subsubsection{Responding in notrump}
\begin{mylist}[.3]
\item Stopper in opponent's suit required
\item 1\NT: 8--11 points
\item 2\NT: 12--14 points after 1-level overcall, 9--11 after 2-level overcall
\item 3\NT: Better than 2\NT\ (or expectation for 9 tricks)
\end{mylist}
Suggests balanced or semibalanced hand and stoppers in unbid suits, but
not required.

\subsection{1NT overcall}
\bullet{15--17 high card points, balanced hand.}
All responses to this (including Stayman, Jacoby, and minor suit busts) are
on, just as if partner had opened the bidding himself.

\subsection{Weak jump overcalls}
\bullet{Long strong suit, up to 10 points}
With a weak hand, overcall preemptively, overbidding by two tricks with
unfavorable vulnerability, three tricks with equal vulnerability, or 
three to five tricks with favorable vulnerability.

\subsubsection{Responses}
\begin{mylist}[0.5]
\item Pass with most hands (even with poor support).
\item Raise partner's suit with mediocre hand and (3+) trump support.
\item Bid own long strong suit if better (independently) than partner's.
\item \NT\ with very strong hand, stoppers in other 3 suits, and good (3+) fit.
\item cue-bid opponent's suit with powerful hand but uncertain of which game.
\end{mylist}
If opponents continue after partner's overcall, may still bid for part
score battle or sacrifice.  Consider how many tricks partner overbid by when
deciding course.\crunch
\except{Jump overcalls after a 1\NT\ opening, after an opening preemptive
bid, or in the balancing seat (without opponent part score) are \emph{strong},
overbidding by at most one trick.}

\eject
\subsection{Unusual 2NT overcall over one of a suit}
\begin{mylist}[.3]
\item Weak or strong: \le12 points or \ge19 points; two-suited (5-5).
\item Shows lowest two unbid suits (over 1\C\to\D\H, 1\D\to\C\H, 1\H\to\C\D, 1\S\to\C\D).
\end{mylist}\crunch[.4]
Consider vulnerability for weak overcalls, and note approximate point ranges.
\subsubsection{Responses}
\begin{mylist}[.3]
\item Takeout to one of partner's suits shows preference, no game interest.
\item Jump takeout to partner's suit shows good support, preemptive or game.
\item Cue-bid enemy suit is game or slam try, forcing.
\item {\small New suit is independent, not forcing; 3\NT\ is to play, 4\NT\ is Blackwood}
\end{mylist}
\subsubsection{Rebids by unusual NT bidder}
\begin{mylist}[.3]
\item Bid or raise one of own suits with extreme shape, natural and not forcing.
\item Strange action (cue-bid, \NT, fourth suit, double) is strong (17+ points).
\item After strange rebid, may sign-off in part-score or game, or encourage.
\end{mylist}
\subsubsection{Other applications}
\begin{mylist}[.3]
\item Over strong two-bid, 2\NT\ is weak two-suiter (note artificial suits unbid).
\item Over 1\NT\ (opening or response), 2\NT\ shows any strength two-suiter.
\item Over any forcing response, 2\NT\ shows any strength two-suiter.
\item Over single major raise, 2\NT\ shows any strength two-suiter.
\item In balancing position (-P-P-2\NT), 2\NT\ is unusual only by a passed hand.
\item[!!] 2\NT\ is \textbf{natural} in balancing position by an unpassed hand.
\item[!!] 1\NT\ in direct position is \textbf{unusual} only by a passed hand.
\item[!!] 3\NT\ is natural unless overcalling 2\NT, or by a passed hand.
\end{mylist}
\subsubsection{Unusual 4NT overcall}
Over a preemptive bid, from an unpassed hand, shows intermediate or better
two-suiter.  Otherwise, implies a weak hand.  All these assume partner has not
bid or doubled, and you have not bid.
\begin{mylist}[.3]\small
\item Over 4\S, shows all unbid suits, or just \H\D\ (clarify later), 13+ points.
\item Over 4\H, shows minor suits, intermediate strength (13+ points).
\item[!!] {\normalsize Over 4\C\D, is \textbf{Blackwood}.}
\item Over lower bid (below 4-level), shows two lower unbid suits.
\end{mylist}
\except{If you have already bid or doubled, and enemy bids 4-of-a-major,
4\NT\ is unusual.  4\S-4\NT\ shows all unbid suits (with strong
preference for bid suit, if applicable).  4\H-4\NT\ shows both minors.}

\subsection{Michaels cue-bid}
\begin{mylist}[.3]
\item Weak or strong: \ltsim12 points or \gtsim19 points; two-suited (5-5).
\item 2\C\D\to both majors, 2\H\S\to other major and an unspecified minor.
\end{mylist}
Michaels cue-bid is a direct cue-bid after a natural opening bid of a suit.
Simplified versions exist (colorful cue-bid shows both opposite-color suits,
higher-suits cue-bid shows the two highest remaining suits).  Cue-bid weaker
hands (\lt 10 points) only with equal or favorable vulnerability.  Note
\emph{approximate} ranges.
\crunch
\except{Over a minor, may show a good 4-card major (3 honors).}

\subsubsection{Responses}
\begin{mylist}[.3]
\item Bid a known suit with preference (3 trumps), no game interest.
\item 2\NT\ after a major asks partner to bid minor.\crunch[.8]
\bullet{also 4\C\ (not forcing) or 4\NT\ (forcing), but 3\NT\ always natural.}
\item Jump in a known suit with good trump support, preemptive (or game).
\item Cue-bid is game or slam try, forcing.
\item {\small New suit is independent, not forcing; 3\NT\ to play; 
2\NT\ after minor invites game.}
\end{mylist}
Always assume cue-bid is weak when responding.  Invite with 9--13 points.\crunch
\except{Commonly (but not standard) 3\D\ after 1\S-2\S\ invites game in hearts.}

\subsubsection{Cue-bidder's rebids}
\begin{mylist}[.3]
\item Bid or raise one of own suits with extreme shape, natural and not forcing.
\item Strange action (cue-bid, \NT, unrevealed 3rd suit (w/ 5530), or X) is strong.
\item {\small After strange rebid, may sign off in suit/\NT, or 4\C (correctable).  Asking for minor w/ 4\NT\ (not Blackwood if minor unknown), or any other bid, is encouraging.}
\end{mylist}

\subsubsection{Other applications}
These assume enemy opened, partner never bid or doubled, cue-bid suit was 
natural, and this is your first opportunity to bid.
\begin{mylist}[.3]
\small
\item Over 1\NT\ response (1\D-P-1\NT-2\D), Michaels is any strength.
\item Over new suit resp. (1\C-P-1\S-2\C), Michaels in \emph{opener's suit} is any strength.
\item[!!] Over new suit response (1\C-P-1\S-2\S), \emph{responder's suit} is \textbf{natural}, strong suit.
\item After Jacoby transfer (1\NT-P-2\D-2\H), Michaels in real suit is any strength.
\item After two passes (1\S-P-P-2\S), Michaels is invitational or better (13+ points).
\item Over preemptive bid (2\S-3\S), Michaels is invitational or better (13+ points).
\item Over a 2-level \emph{non-preempt} (1\H-P-2\H-3\H), Michaels is usually weak (\ltsim12 points).
\item From a passed hand (P-1\C-P-1\NT-2\C), Michaels is always weak (\le 12 points).
\end{mylist}




\section{Takeout Doubles}






%\begin{minipage}[t]{0.5\columnwidth}
%For the opener,\\
%\begin{tabular}{|l|l|}\hline
%13--16 & minimum opening\\\hline
%17--19 & invitational\\\hline
%20--22 & game strength\\\hline
%23+    & slam strength\\\hline
%\end{tabular}
%\end{minipage}
%\begin{minipage}[t]{0.5\columnwidth}
%For the responder,\\
%\begin{tabular}{|l|l|}\hline
%6--10  & minimum response\\\hline
%11--12 & invitational\\\hline
%13--19 & game strength \\\hline
%20+    & slam strength\\\hline
%\end{tabular}
%\end{minipage}


\end{document}
