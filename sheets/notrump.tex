\begin{sheet}{Responses to a notrump opening}\setcounter{section}0

\noindent The notrump opening is a limit bids.  Responder is the captain
and ultimately decides where to end up.

\begin{columns}{3}
\begin{column}
\section{Point Ranges}
Responding to 1\nt\ (partner has 15--17),
these are the important ranges:
\begin{description}\itemsep-4pt
\item[0--7]sign off
\item[8--9]invitational
\item[10--15]game-forcing
\item[16--17]slam-invitational
\item[18+]slam-forcing
\end{description}
Decrease numbers by 5 points when
responding to 2\nt\ (20--22).
\end{column}

\begin{column}
\section{Bids}
\begin{description}\itemsep-4pt
\item[2\c]Stayman
\item[2\d/2\h]Jacoby transfer
\item[2\s/2\nt]Minor suit transfer
\item[3\c--3\s]?
\item[3\nt]Sign off
\item[4\c]Gerber
\item[4\d/4\h]Texas transfer
\end{description}
Stayman, Jacoby, Gerber, and Texas are all
on after 2\nt\ as well, only shifted appropriately.
\end{column}

\begin{column}
\section{Stayman}
Stayman asks opener to show a 4-card major suit.  It can be bid with
any strength and any shape.  Opener responds
2\d\ with no 4-card majors, 2\h\ with 4 hearts (and maybe 4 spades),
and 2\s\ with 4 spades (and not 4 hearts).  Responder's rebids are as follows:
\begin{description}\itemsep-4pt
\item[\pass/2\h]Sign off with nothing
\item[2\s/2\nt]Invite game in \s\ and/or \nt
\item[3 same major]Invite game in the major
\item[3\nt]Sign off, denying fit
\end{description}
\end{column}

\end{columns}


\iffalse
{\strut\hfill
\begin{ptable}\headings{0-5|6-8|9-12|13|14|15|16|17|18|19|20|21|22|23+}
\row{balanced}
  \bid(13-14,lgreen){}
  \bid(15-17,lyellow){1\nt}
  \bid(18-19,lgreen){}
  \bid(20-22,lred){2\nt}
  \bid*(23,lpurple){2\c}\\
\row{openable suit}%\footnote{Opening a suit requires 5 of a major or 4 of a minor}}
  \bid*(13-22,lgreen){1\c\d\h\s}
  \bid*(23,lpurple){2\c}\\
\row{6 card suit}
  \bid(6-12,lblue){2\d\h\s}
  \bid(13-22,lgreen){1\c\d\h\s}
  \bid(23,lpurple){\clift{2\c\w{a}}}\\
\row{7 card suit}
  \bid(0-8,lorange){\ge3\c\d\h\s}
  \bid*(9-12,lblue){2\d\h\s}
  \bid*(13-22,lgreen){1\c\d\h\s}
  \bid*(23,lpurple){2\c}\\
\spanrow{5}{solid minor, 2 outside stoppers}%\footnote{7-card minor (sometimes 6), usually all 3 outside suits stopped, never a singleton or void}}
  \bid(16-21,laqua){Acol 3\nt\w{a}}\\
\end{ptable}\hfill\strut}

%\raggedright
%\rightskip=0pt plus 1fil

\begin{columns}8
%\expandspaces{0.5}
%\sloppy
\begin{column}[span=3]%[flow=2]
\section{Suit openings}
The basic suit openings are 1\c, 1\d, 1\h, and 1\s.  All four openings have
the same strength requirements: between 13 and 22 points, including
distribution.  With very long suits, the upper limit is lower.
\subsection{Five-card majors}
In order to more easily find a 5-3 fit, we require a 5-card major suit
to open a major (1\h\ or 1\s).
\subsection{Better minor}
Always open a major if possible.  With no 5-card major, open the longest
minor suit (this may be as short as 3 cards).  It is conventional to open
1\c\ with 3-3 and 1\d\ with 4-4 in the minors.
\end{column}

\begin{column}[span=5]
\section{Notrump openings}
With a balanced hand (5332, 4432, or 4333) and opening strength, you should
think about opening notrump.  The point ranges are narrow (3 points each)
and only count high card points.  With 15--17 points, open 1\nt, and with
20--22 points, open 2\nt.  Note that 3\nt\ is a conventional opening (Acol)
and is described later.

\section{Preemptive openings}
With a weak hand (definitely less than opening strength) and a long suit
(6 or more), you should make a preemptive opening.
\subsection{Weak two-bids}
A bid at the 2-level (other than 2\c, which is reserved for strong hands)
shows a 6-card suit and roughly 5--9 high card points.  With any more high
card points, the distribution should be enough to open, so a preempt should
be avoided.
\subsection{Higher preempts}
With more than 6 cards in a suit, increase the level of the bid by one
for each extra card.  This can be done with any strength, but with stronger
hands, a 1- or 2-level opening may be preferred.
\end{column}
\end{columns}
\fi

\end{sheet}
