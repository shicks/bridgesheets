% \iffalse
%
% bridge.dtx - Created 7/30/2008 Steve Hicks
%
% Run LaTeX on this document to produce documentation.
% Run LaTeX on the .ins file to produce the package.
%<*driver>
\ProvidesFile{bridge.dtx}
%</driver>
%
%<package>\NeedsTeXFormat{LaTeX2e}
%<package>\ProvidesPackage{bridge}%
           [2008/07/30 v1.0 Bridge package]
%<*driver>
\documentclass{ltxdoc}
%\CheckSum{}
%\OnlyDescription % (un)comment this line to show (hide) source code
\RecordChanges
\EnableCrossrefs
\CodelineIndex   % (un)comment this line to index source by page (line)
\begin{document}
  \newcommand*{\Lopt}[1]{\textsf {#1}}
  \parindent0pt
  \def\*#1{\texttt{\string#1}} %% sdh - |...| doesn't work in headings
  \makeatletter
  %
  \newcount\mac@depth\mac@depth\z@
  \newcommand\@macros{}\newcommand\@endmacros{}
  \catcode`&3  %% we use a funny catcode to ensure never used.
  \def\@macros#1,{\macro{#1}\global\advance\mac@depth\@ne\relax
    \@ifnextchar&\@gobble\@macros}
  \def\@endmacros{\let\mac@next\relax\ifnum\mac@depth>\z@
    \endmacro\let\mac@next\@endmacros
    \global\advance\mac@depth\m@ne\fi\mac@next}
  \newenvironment{macros}[1]{\@macros#1,&}{\@endmacros}
  \catcode`&4  %% put it back 
  \makeatother %% must be balanced for character table to work properly
  %
  \DocInput{bridge.dtx}
  \setcounter{IndexColumns}{2}
  \PrintIndex
  \PrintChanges
\end{document}
%</driver>
% \fi
% \changes{v1.0}{2008/07/30}
%       {(SDH) Initial version.}
%
% \GetFileInfo{bridge.dtx}
% \title{|bridge| package documentation}
% \author{Stephen Hicks}
% \date{\fileversion{} -- \filedate}
% \maketitle
%
% \part*{Usage}
%
% \StopEventually{}
%
% \makeatletter
% \part*{Implementation}
% Make the |@|-sign into a letter for use in macro names.
%    \begin{macrocode}
%<*package>
\makeatletter
%    \end{macrocode}
%
% \begin{macros}{\nlet,\rlet,\ndef}
% We define a few commands to take advantage of \LaTeX's safety features.
% These behave exactly like the \TeX{} counterparts, but they use
% |\newcommand| or |\renewcommand| first to make sure that we're not
% (or are) clobbering anything.
%    \begin{macrocode}
\newcommand\nlet[2]{%
  \newcommand#1{}\let#1#2%
}
\newcommand\rlet[2]{%
  \renewcommand#1{}\let#1#2%
}
\newcommand\ndef[1]{%
  \newcommand#1{}\def#1%
}
%    \end{macrocode}
% \end{macros}
%
% \subsection{Option processing}
%
% \begin{macro}{\if@BRcolor}
% The \Lopt{color} option sets all the suits in solid colors.
%    \begin{macrocode}
\newif\if@BRcolor
\DeclareOption{color}{\@BRcolortrue}
%    \end{macrocode}
% \end{macro}
%
% \begin{macro}{\if@BRfourcolor}
% The \Lopt{fourcolor} option sets the suits in four different colors.
% This automatically sets \Lopt{color}.
%    \begin{macrocode}
\newif\if@BRfourcolor
\DeclareOption{fourcolor}{\@BRfourcolortrue\@BRcolortrue}
%    \end{macrocode}
% \end{macro}
%
% Finally, process the options.
%    \begin{macrocode}
\ProcessOptions\relax
%    \end{macrocode}
%
% \subsection{Package requirements}
% We load |color| and |pifont| if needed.
%    \begin{macrocode}
\if@BRcolor
  \RequirePackage{color}
  \RequirePackage{pifont} % needed for solid heart/diamond
\fi

 % These are imported from the old version...
\RequirePackage{colortbl}
\RequirePackage{calc,graphicx}


%    \end{macrocode}
%
% \section{Suit setup}
% \begin{macros}{\c,\d,\h,\s,\nt}
% We define macros |\c|, |\d|, |\h|, |\s|, and |\nt|.
%    \begin{macrocode}
\if@BRcolor
  \renewcommand\c{\textcolor{clubcolor}{\Pisymbol{psy}{"A7}}}
  \renewcommand\d{\textcolor{diamondcolor}{\Pisymbol{psy}{"A8}}}
  \newcommand\h{\textcolor{heartcolor}{\Pisymbol{psy}{"A9}}}
  \newcommand\s{\textcolor{spadecolor}{\Pisymbol{psy}{"AA}}}
  \newcommand\nt{\textcolor{ntcolor}{\textsc{nt}}}
\else
  \renewcommand\c{$\clubsuit$}
  \renewcommand\d{$diamondsuit$}
  \newcommand\h{$\heartsuit$}
  \newcommand\s{$\spadesuit$}
  \newcommand\nt{\textsc{nt}}
\fi
%    \end{macrocode}
% \end{macros}
%
% \subsection{Colors}
% \begin{macros}{clubcolor,diamondcolor,heartcolor,spadecolor}
% We need to set the colors as well.
%    \begin{macrocode}
\if@BRcolor
  \if@BRfourcolor
    \definecolor{clubcolor}{rgb}{0,0.35,0.45}
    \definecolor{diamondcolor}{rgb}{0.6,0.6,0}
    \definecolor{heartcolor}{rgb}{0.8,0,0}
    \definecolor{spadecolor}{rgb}{0,0,0}
    \definecolor{ntcolor}{rgb}{0,0,0}
  \else
    \definecolor{clubcolor}{rgb}{0,0,0}
    \definecolor{diamondcolor}{rgb}{0.8,0,0}
    \definecolor{heartcolor}{rgb}{0.8,0,0}
    \definecolor{spadecolor}{rgb}{0,0,0}
    \definecolor{ntcolor}{rgb}{0,0,0}
  \fi
\fi
%    \end{macrocode}
% \end{macros}
%
% \section{Tables}
% We have some fancy macros for setting up point range tables.
%    \begin{macrocode}

 % Here are a bunch of routines for the ptable environment
 % The basic syntax is
 % \begin{ptable}\headings{#-#|#-#|...|#+}
 %   \row{row heading}
 %     \bid(#-#,color){text}
 %     \bid*(#-,color){text}\\ % to end of table, no text
 %   \row{...}
 %     \bid...
 % \end{ptable}
 % Also commands \drop[dimension]{text}, \clift[dimension]{text}
 % for row headings and bids, respectively
 % \brace{skip}{height} puts a brace ahead a certain distance

\let\xa\expandafter
\newcount\c@a\newcount\c@b\newcount\c@c
\newif\ifplus
\def\@gobble@null\@null{}
\def\@maybe@next{\ifx\@foo\@null\let\@next\@gobble@null\fi\@next}
\def\@headings#1|{\doheading#1-.-\@end\let\@next\@headings\futurelet\@foo\@maybe@next}
\def\headings#1{\global\c@c\z@\@headings#1|\@null\\\hline\hline}
\def\stripplus#1+#2\@null{\ifx+#2\relax\plustrue\else\plusfalse\fi\c@a#1\relax}
\def\setupper#1#2\@null{\ifx.#1\relax\number\c@a\relax\c@b\c@a\relax\else\c@b#1#2\relax\number\c@a--\number\c@b\fi\ifplus+\fi}
\def\doheading#1-#2-#3\@end{%
  &%  \ifnum\c@c>\z@&\fi
  \stripplus#1+\@null % sets \ifplus and \c@a
  \setupper#2\@null   % sets \c@b and prints stuff
  \xa\xdef\csname a@\romannumeral\c@a\endcsname{\number\c@c}%
  \xa\xdef\csname b@\romannumeral\c@b\endcsname{\number\c@c}%
  \ifplus\xdef\b@plus{\number\c@b}\fi
  \global\advance\c@c\@ne % next column...
}

\def\getleft{\xa\c@a\csname a@\romannumeral\c@a\endcsname}
\def\getright{\xa\c@b\csname b@\romannumeral\c@b\endcsname}

\def\@@lign{%
  \t@b%  % This should work now...
  \let\@next\@@lign
  \ifnum\c@c<\c@a
    \global\advance\c@c\@ne
  \else
    \let\@next\relax
  \fi\@next}

\newcount\@left
\newcount\@cur

\def\t@b{&}
\def\@lign{%
  \global\@left\c@a % copy this value
  \global\advance\@left-\@cur % find out how many more we need to go.
  \ifnum\@left>\z@  % SOMEHOW, this counter is getting changed on me!!!
    \t@b\xa\multicolumn\xa{\number\@left}{c}{}\t@b%
  \else\t@b\fi
}

\def\@bid#1#2#3#4{%
  %\message{#1-#2}%
  \c@a#1\relax\getleft % \c@c stores the current position in the row...
  \c@b#2\relax\getright\advance\c@b\@ne
  \global\@cur\c@c\relax\global\c@c\c@b\relax % make sure to update it...
  \global\advance\c@b-\c@a\relax
  \@lign\mcc{\number\c@b}{#3}{#4}%
}
 %\def\resolverange{% First check for a + after first bid
 %  \xa\@resolverange\fix@@plus}
 %\def\fix@@plus{}%
 %\def\fix@plus#1{%
 %  \let\@next\fix@plus
 %  \ifx#1-\relax\let\@next\relax\fi
 %  \ifx#1+\relax-\b@plus\let\@next\relax
 %  \else#1\fi
 %  \@next
 %}
\def\@pick#1#2{\ifx\@null#1\@null#2\else#1\fi} % puts #2 is #1 is empty
\def\@resolverange#1-#2-#3\@null{{#1}{\@pick{#2}{\b@plus}}}
\def\@@bid#1#2#3{% #1 is a range
\xa\@bid\@resolverange#1-#1-\@null{#2}{#3}}

\newif\if@star
\newif\if@usingstar
\global\@usingstartrue
\def\bid{\futurelet\@foo\check@star}
\def\check@star{\let\@next\@nsbid
  \ifx\@foo*\relax
    \if@usingstar\global\@startrue\fi
 %    \global\let\if@star\if@usingstar % bad syntax...
    \let\@next\@sbid
  \else
    \global\@starfalse
  \fi\@next}
\def\@sbid*{\@nsbid}
\def\@nsbid(#1,#2)#3{\@@bid{#1}{#2}{\if@star\else#3\fi}}

\def\clear@ranges{\count@40\let\@next\@clr@ranges\@next}
\def\@clr@ranges{\ifnum\count@<\z@\let\@next\relax\else
  \xa\global\xa\let\csname a@\romannumeral\count@\endcsname\undef
  \xa\global\xa\let\csname b@\romannumeral\count@\endcsname\undef
  \global\advance\count@\m@ne
\fi\@next}

\def\b@ptable{\begin{tabular}{lccccccccccccccccccccc}} % should be enough...
\def\e@ptable{\hline\end{tabular}\clear@ranges}

\def\gap{\\[-.6pc]}
\newcommand{\drop}[2][0.52pc]{\setbox0\hbox{\lower #1\hbox{#2}}\dp0=0pt\ht0=0pt\box0}
\newcommand{\clift}[2][0.45pc]{\if@usingstar\setbox0\hbox{\raise #1\hbox{#2}}\dp0=0pt\ht0=0pt\box0\else#2\fi} % conditional lift - only if using stars

\newenvironment{ptable}{\b@ptable}{\e@ptable}

\def\row{\global\c@c\z@}
\def\spanrow#1#2{\multicolumn{#1}{l}{#2}\c@c\m@ne\global\advance\c@c#1}
\def\ob#1#2{\@@bid{#1}{lorange}{#2}}
\def\yb#1#2{\@@bid{#1}{lyellow}{#2}}
\def\gb#1#2{\@@bid{#1}{lgreen}{#2}}
\def\bb#1#2{\@@bid{#1}{lblue}{#2}}
\def\pb#1#2{\@@bid{#1}{lred}{#2}}
\def\vb#1#2{\@@bid{#1}{lpurple}{#2}}
\def\ab#1#2{\@@bid{#1}{laqua}{#2}}

 % There MUST be better ways to do this...
\def\oc#1#2{\@@bid{#1}{lorange}{}}
\def\yc#1#2{\@@bid{#1}{lyellow}{}}
\def\gc#1#2{\@@bid{#1}{lgreen}{}}
\def\bc#1#2{\@@bid{#1}{lblue}{}}
\def\pc#1#2{\@@bid{#1}{lred}{}}
\def\vc#1#2{\@@bid{#1}{lpurple}{}}
\def\ac#1#2{\@@bid{#1}{laqua}{}}

\def\ic{\futurelet\@foo\@ic}
\def\@ic{\let\@next\g@ic\ifx\@foo\egroup\let\@next\relax\fi\@next}
\def\g@ic#1{\futurelet\@foo\@ic}

\def\ditto#1{\raise 0.18pc\hbox to #1{\leaders\hrule\hfill\lower 0.4pc\hbox{\kern1ex''\kern0.6ex}\leaders\hrule\hfill}}

 %\def\savenote#1#2{\begingroup\count@\value{footnote}\advance\count@\@ne\xdef#1{\the\count@}\endgroup\footnote{#2}}
 %\def\savenote#1#2{\footnote{#2}\xdef#1{\value{footnote}}}

 % Want some way to signify various properties - forcing, overcalls, etc...
\def\w#1{\def\@top{}\def\@bot{}\@w#1\@go}
\let\ww\w
\def\@w#1{\let\@next\@w\ifx#1\@go\relax\let\@next\@go\else
  \ifx#1o\relax\edef\@top{\@top o}\fi
  \ifx#1d\relax\edef\@top{\@top d}\fi
  \ifx#1a\relax\edef\@bot{\@bot a}\fi
  \ifx#1i\relax\edef\@bot{\@bot i}\fi
  \ifx#1s\relax\edef\@bot{\@bot s}\fi
  \ifx#1f\relax\edef\@bot{\@bot f}\fi
  \ifx#1g\relax\edef\@bot{\@bot g}\fi
\fi\@next}
\def\@go{\kern .2em\smash{\rlap{\raise 0.8ex\hbox{\tiny\@bot}}\rlap{\lower 0.2ex\hbox{\tiny\@top}}}}


\def\brace#1#2{\setbox0=\hbox{\hskip #1$\left\{\vbox to #2{}\right.$}%
  \wd0=0pt\ht0=0pt\dp0=0pt\box0}


 % Defines command #1 which calls #2 with color in #3
\def\colorcommand#1#2#3{%
  \newcommand{#1}[2][\no@arg]{%
    \ifx##1\no@arg\relax\def#3{{##2}}%
    \else\def#3{[##1]{##2}}\fi#2}}

\newcommand{\@highlight}[2][2pt]{\begingroup
  % Make a colorbox to get padding - \@hl@col is the color (w/ braces)
  \def\@cb{\expandafter\colorbox\@hl@col}%
  \def\x{\dimen\z@}\def\y{\dimen\@ne}\def\z{\dimen\tw@}\def\p{\dimen\thr@@}%
  \def\w{\wd\z@}\def\h{\ht\z@}\def\d{\dp\z@}%
  \setbox\z@\hbox{\@cb{}}%
  \p#1\relax\multiply\p\tw@ % twice the padding
  % \dimen0 is the highlighting width, \dimen1 is the height+depth
  \x-\w \y-\h \advance\y-\d
  \advance\x\p \advance\y\p
  % \dimen2 is just the depth
  \z-\d \advance\z#1\relax
  % Now set the text to get its dimensions
  \setbox\z@\hbox{#2}%
  \advance\x\w \advance\z\d
  \advance\y\h \advance\y\d
  % Set the actual colorbox
  \setbox\z@\hbox{\hskip-#1\@cb{\hbox to \x{\vbox to \y{}}}}%
  % Clear out its dimens and draw it (lowered)
  \w=0\p@\h=0\p@\d=0\p@\lower\z\box\z@\endgroup\hbox{#2}}
\colorcommand\highlight\@highlight\@hl@col

\def\mcc#1#2#3{\multicolumn{#1}{>{\columncolor{#2}}c}{#3}}
\definecolor{lblue}{rgb}{.75,.8,1}
\definecolor{lred}{rgb}{1,.8,.8}
\definecolor{lgreen}{rgb}{.8,1,.8}
\definecolor{lorange}{rgb}{1,.9,.7}
\definecolor{lpurple}{rgb}{.95,.7,1}
\definecolor{lyellow}{rgb}{1,1,.6}
\definecolor{laqua}{rgb}{.65,.95,1}

%    \end{macrocode}
%
% \section{End}
%    \begin{macrocode}
\makeatother
%</package>
%    \end{macrocode}

% \makeatother
% \Finale
%
%
% \iffalse
%
% The next line of code prevents DocStrip from adding the
% character table to the generated files(s).

% Removed stuff

\endinput
%
% \fi
%
%% \CharacterTable
%%  {Upper-case    \A\B\C\D\E\F\G\H\I\J\K\L\M\N\O\P\Q\R\S\T\U\V\W\X\Y\Z
%%   Lower-case    \a\b\c\d\e\f\g\h\i\j\k\l\m\n\o\p\q\r\s\t\u\v\w\x\y\z
%%   Digits        \0\1\2\3\4\5\6\7\8\9
%%   Exclamation   \!     Double quote  \"     Hash (number) \#
%%   Dollar        \$     Percent       \%     Ampersand     \&
%%   Acute accent  \'     Left paren    \(     Right paren   \)
%%   Asterisk      \*     Plus          \+     Comma         \,
%%   Minus         \-     Point         \.     Solidus       \/
%%   Colon         \:     Semicolon     \;     Less than     \<
%%   Equals        \=     Greater than  \>     Question mark \?
%%   Commercial at \@     Left bracket  \[     Backslash     \\
%%   Right bracket \]     Circumflex    \^     Underscore    \_
%%   Grave accent  \`     Left brace    \{     Vertical bar  \|
%%   Right brace   \}     Tilde         \~}
%%
