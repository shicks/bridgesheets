
\begin{sheet}{Openings}

{\strut\hfill
\begin{ptable}\headings{0-5|6-8|9-12|13|14|15|16|17|18|19|20|21|22|23+}
\row{balanced}
  \bid(13-14,lgreen){}
  \bid(15-17,lyellow){1\nt}
  \bid(18-19,lgreen){}
  \bid(20-22,lred){2\nt}
  \bid*(23,lpurple){2\c}\\
\row{openable suit}%\footnote{Opening a suit requires 5 of a major or 4 of a minor}}
  \bid*(13-22,lgreen){1\c\d\h\s}
  \bid*(23,lpurple){2\c}\\
\row{6 card suit}
  \bid(6-12,lblue){2\d\h\s}
  \bid(13-22,lgreen){1\c\d\h\s}
  \bid(23,lpurple){\clift{2\c\w{a}}}\\
\row{7 card suit}
  \bid(0-8,lorange){\ge3\c\d\h\s}
  \bid*(9-12,lblue){2\d\h\s}
  \bid*(13-22,lgreen){1\c\d\h\s}
  \bid*(23,lpurple){2\c}\\
\spanrow{5}{solid minor, 2 outside stoppers}%\footnote{7-card minor (sometimes 6), usually all 3 outside suits stopped, never a singleton or void}}
  \bid(16-21,laqua){Acol 3\nt\w{a}}\\
\end{ptable}\hfill\strut}

%\raggedright
%\rightskip=0pt plus 1fil

\begin{columns}8
%\expandspaces{0.5}
%\sloppy
\begin{column}[span=3]%[flow=2]
\section{Suit openings}
The basic suit openings are 1\c, 1\d, 1\h, and 1\s.  All four openings have
the same strength requirements: between 13 and 22 points, including
distribution.  With very long suits, the upper limit is lower.
\subsection{Five-card majors}
In order to more easily find a 5-3 fit, we require a 5-card major suit
to open a major (1\h\ or 1\s).
\subsection{Better minor}
Always open a major if possible.  With no 5-card major, open the longest
minor suit (this may be as short as 3 cards).  It is conventional to open
1\c\ with 3-3 and 1\d\ with 4-4 in the minors.
\end{column}

\begin{column}[span=5]
\section{Notrump openings}
With a balanced hand (5332, 4432, or 4333) and opening strength, you should
think about opening notrump.  The point ranges are narrow (3 points each)
and only count high card points.  With 15--17 points, open 1\nt, and with
20--22 points, open 2\nt.  Note that 3\nt\ is a conventional opening (Acol)
and is described later.

\section{Preemptive openings}
With a weak hand (definitely less than opening strength) and a long suit
(6 or more), you should make a preemptive opening.
\subsection{Weak two-bids}
A bid at the 2-level (other than 2\c, which is reserved for strong hands)
shows a 6-card suit and roughly 5--9 high card points.  With any more high
card points, the distribution should be enough to open, so a preempt should
be avoided.
\subsection{Higher preempts}
With more than 6 cards in a suit, increase the level of the bid by one
for each extra card.  This can be done with any strength, but with stronger
hands, a 1- or 2-level opening may be preferred.
\end{column}
\end{columns}

\begin{columns}{11}

\begin{column}[span=6]
\section*{Strong 2\c\ opening}
With a balanced hand and 23 points, open 2\c.  With very long suits,
you may open 2\c\ with a few less points.

\section*{Acol 3\nt\ opening}
Open a gambling 3\nt\ with a solid 6--7 card minor suit, 16--21 points,
and at least two outside stoppers.
\end{column}

\begin{column}\strut\end{column}

\begin{column}[span=4]
\setbox0\vbox{\fixcolumnwidth
\section*{3rd seat: Rule of twenty}
In third seat, open if (high card points + length of two longest suits) \ge\ 20.

\section*{4th seat: Rule of fifteen}
In third seat, open if (high card points + length of spades) \ge\ 15.
}\dp0=0pt\ht0=0pt\vskip1pc\box0
\end{column}
\end{columns}

\vskip .2in
\begin{columns}1\begin{column}\strut\hfill\Huge Basics\hfill\strut
\end{column}\end{columns}\setcounter{section}{0}

\begin{columns}3\begin{column}
\section*{Suit order}
\c\to\d\to\h\to\s\to\nt

\section*{Point count}
A=4, K=3, Q=2, J=1\\
void=3, singleton=2, doubleton=1

\subsection{Unprotected honors}
K, Qx, Jxx are unprotected.  Count \emph{either} high card points
or distribution but not both.
\end{column}

\begin{column}
%\makeatletter
%\vbox to 1.8in{\vfill\hbox to \BS@colwidth{\hfill\HUGE ?\hfill}\vfill}
%\makeatother
\section*{Game bids}
$\displaystyle
 \left.\begin{array}{l}3\nt\\\hbox{4\h, 4\s}
 \end{array}\right\}\hbox{ 26 points}$\hfill\\
\strut\hskip1.2ex 5\c, 5\d: 29 points
\section*{Slam bids}
\strut\hskip1.2ex 6-level: 33 points\\
\strut\hskip1.2ex 7-level: 37 points
\section*{Law of total tricks}
In competition, bid to take as many tricks as there are trump
between the partners.
\end{column}

\begin{column}%[flow]
\section*{Scoring}
\c,\d\ contract: 20/trick\\
\h,\s\ contract: 30/trick\\
\nt\ contract: 40/first, 30/after\\
%\\[-.5pc]
undertricks: 50 (100 if vulnerable)\\
\strut\hskip1em dbl'd: 1/2/2/300 (2/300 vuln)\\
%\\[-.5pc]
doubled contract made: +50\\
%\\[-.5pc]
small slam: 500 (750 vuln)\\
grand slam: 1000 (1500 vuln)\\
%\\[-.5pc]
rubber: 700 (in 2), 500 (in 3)\\
%\\[-.5pc]
at end: game=300, part=100
 % There is some weirdness going on with skips...!
 % (of course... just need to redefine the section/subsection skips...)
 % (and stop overriding the real vskip!)
\end{column}

\end{columns}

\end{sheet}
